\documentclass[KHI_DAN_HOI.tex]{subfiles}
\begin{document}
\chapter{GIỚI THIỆU VỀ KHÍ ĐÀN HỒI}
\section{Định nghĩa}
Khái niệm \emph{khí đàn hồi} đã được áp dụng bởi các kỹ sư hàng không vào một lớp quan trọng các bài toán trong thiết kế máy bay. Nó thường được định nghĩa như là một môn khoa học nghiên cứu các tương tác qua lại giữa các lực khí động là lực đàn hồi, và tác động của các tương tác này lên việc thiết kế máy bay. Các bài toán khí đàn hồi sẽ không tồn tại nếu một chiếc máy bay hoàn toàn cứng. Các kết cấu máy bay hiện đại rất linh hoạt, và tính linh động là chịu trách nhiệm chính cho rất nhiều hiện tượng khí đàn hồi. Tính linh động cấu trúc tự bản thân nó không thể bị phản đối; tuy nhiên, các hiện tượng khí đàn hồi sinh ra khi có sư biến dạng cấu trúc cảm sinh thêm lực khí động. Các lực khí động phụ thêm này 

\end{document}