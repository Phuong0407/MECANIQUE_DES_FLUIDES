\documentclass[13.5pt,twoside,a4paper]{extbook}
\usepackage[inner=2cm,outer=2cm,top=2cm,bottom=2cm]{geometry}
\usepackage{amssymb,amsmath,amsthm,mathtools}
\usepackage[utf8]{vietnam}
\usepackage{indentfirst}
\usepackage{framed}

\begin{document}
\chapter{LÝ THUYẾT CHONG CHÓNG}
\section{Lý thuyết động lượng}

Các giả thiết được áp dụng:
\begin{enumerate}
    \item Chất lưu chảy hoàn hảo, ổn định và không nén.
    \item Mô hình đĩa truyền động được áp dụng cho cánh turbine và đĩa lấy năng lượng từ dòng chảy.
    \item Đĩa truyền động tạo ra sự gián đoạn áp suất.
    \item Dòng chảy là đều dọc theo đĩa và trong vết hậu lưu.
    \item Đĩa không truyền bất kỳ xoáy nào đối với dòng chảy. Ảnh hưởng của vết hậu lưu xoay sẽ được bổ sung sau trong chương này.
\end{enumerate}

\section{Lý thuyết động lượng với hậu lưu xoay}
Sự phân tích động lượng được điều chỉnh để cho phép đĩa truyền động truyền sự xoay cho dòng chảy ở phía sau đĩa.

Dòng chảy ở phía trước đĩa không bị ảnh hưởng. Ở ngay phía sau đĩa truyền động, dòng chảy tiếp tuyến được truyền cho hậu lưu.

Chúng ta biểu diển dòng chảy tiếp tuyến bởi một tham số là tham số cảm úng góc (angular induction factor) $b$, trong đó :
$$
b=\frac{\omega}{\Omega}
$$
trong đó $\omega$ là vận tốc góc được truyền cho hậu lưu và $\Omega$ là vận tốc góc của đĩa truyền động. Chúng ta sẽ giả thiết là $b$ không đáng kể so với 1.

Từ mô hình Glauert, lực đẩy nguyên tố trên một hình vành khăn của đĩa truyền động được tính:
\[dT = \Delta p2\left( {2\pi rdr} \right) = \left( {\rho \left( {\Omega  + \frac{\omega }{2}} \right)\omega {r^2}} \right)2\pi rdr\]

Khi sử dụng tham số $b$, chúng ta có :








\end{document}
