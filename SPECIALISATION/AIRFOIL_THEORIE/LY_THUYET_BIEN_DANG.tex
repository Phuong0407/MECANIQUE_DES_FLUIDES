\documentclass[a4paper]{extbook}

\usepackage[inner=2cm,outer=1.5cm,top=1.5cm,bottom=1.5cm]{geometry}
\usepackage{subfiles,graphicx,indentfirst,scrextend}
\usepackage{mathtools,mathrsfs,amssymb,amsthm,amsmath,bbm}
\usepackage[utf8]{vietnam,inputenc}

\changefontsizes{13pt}

\begin{document}
\chapter{Ý NGHĨA CỦA CÁC ĐẶC TÍNH CỦA BIÊN DẠNG CÁNH}

\section{Mở đầu về khí động của biên dạng cánh}

Trong nghiên cứu này, dòng chuyển động qua cánh có thành phần chủ yếu là theo phương dây cung cánh và số Mach là dưới số Mach tới hạn, chỉ trong trường hợp này ta mới có thể áp dụng các tính chất của diện tích mặt cắt ngang của cánh để đoán các đặc tính của cánh.

Ở đây ta nêu ra một số đặc tính quan trọng của biên dạng cánh để làm cơ sở cho các nghiên cứu tiếp theo. Hệ số lực nầng tăng gần như là tuyến tính theo góc tới và đồ thị hệ số lực cản theo lực nâng là một đường cong gần như là parabol cho đến khi có hiệu tượng stall. Hiện tượng stall là hiện tượng trong đó hệ số lực nâng đạt đến giá trị cực đại mà nó có thể đạt được.

Một cánh được thiết kế là tốt nếu như lực cản là càng nhỏ so với lực nâng. Điều này có thể thực hiện bằng cách giảm diện tích cánh, nhưng diện tích cánh không thể giảm một cách tùy tiện mà chỉ có thể giảm sao cho không gây ra hiện tượng stall. Ta muốn một máy bay bay bằng phải có hệ số lực nâng cao kết hợp vớp một hệ số lực cản nhỏ. Đối với các máy bay không đạt được điều này, ta có thể sử dụng các bề mặt điều khiển để tăng lực nâng.

\subsection{Ảnh hưởng của tỉ lệ bình diện}

Tỉ lệ bình diện càng lớn thì có độ dốc đường lực nâng càng lớn và một hệ số lực cản nhỏ. 

Đối với một cánh có tỉ lệ bình diện sao cho sự phân bố lực nâng là ellip dọc theo chiều dài sải cánh thì sự liên hệ như sau giữa hệ số lực nâng và hệ số lực cản:
\begin{align}
    C_D = C_{D_0} + \dfrac{C_L^2}{\pi AR},
\end{align}
trong đó $C_{D_0}$ là hệ số lực cản khi tỉ lệ bình diện cánh là vô cùng - tức là khi bài toán thu về bài toán hai chiều.

Tương tự, góc tấn của cánh phải có liên hệ với góc tấn của cánh có tỉ lệ bình diện vô cực để có cùng lực nâng:
\begin{align}
    \alpha' = \alpha + \dfrac{C_L}{\pi AR},
\end{align}

\section{Áp dụng các đặc tính của biên dạng cho cánh đơn}

Để áp dụng lý thuyết này, chúng ta sẽ sữ dụng lý thuyét đường lực nâng. 






\end{document}