\documentclass[KHI_DONG_HOC.tex]{subfiles}

\begin{document}

\chapter{LÝ THUYẾT CÁNH}

        Chúng ta đã thấy rằng lực nâng của của biên dạng cánh liên kết với sự tồn tại của một lưu số khác không xung quanh biên dạng cánh. Mà sự tồn tại của lưu số này được được là do xoáy khởi động được tạo ra khi dòng chảy ngược dòng đi quan biên dạng cánh: lưu số xung quanh biên dạng bằng độ lớn và ngược chiều với lưu số của xoáy khởi động và trong chế độ ổn định được xác định bởi điều kiện Kutta.

        Điều gì xảy ra với biểu diễn này khi sải cánh hữu hạn ? Để đơn giản, chúng ta giả định rằng biên dạng cánh vẫn còn mỏng: sải cánh của nó lớn hơn nhiều so với dây cung cánh ở gốc và các đặc điểm hình học của cấu hình thay đổi chậm theo hướng sải cánh. Theo định luật Helmholtz (chương I, tiết 9), các sợi xoáy cấu tạo nên lớp xoáy liên kết với cánh không thể kết thúc trong chất lưu: chúng là những đường cong khép kín hoặc những đường có độ dài vô hạn. Bạn phải dựa vào thí nghiệm để biết thêm thông tin. Hình ảnh hóa dòng chảy xung quanh một cánh hình chữ nhật tiết lộ các xoáy cuối có trục, từ rất xấp xỉ, song song với hướng của vận tốc ngược dòng vô hạn (hình 10.2).

        Lời giải thích cho hiện tượng này rất đơn giản: theo lý thuyết hai chiều, áp suất trung bình ở mặt trên thấp hơn so với mặt dưới. Sự chênh lệch áp suất này tạo ra một dòng chảy thứ cấp xung quanh mũi cánh có xu hướng thay thế các đường dòng hướng về phía gốc cánh ở mặt trên và về phía mũi cánh ở mặt dưới.

        Do đó, vận tốc liên quan đến dòng thứ cấp này không liên tục bởi hai bên của cánh và sự gián đoạn này được duy trì ở phần sau. Tính đến các kết quả chung được nêu trong ..., bước nhảy tốc độ nhất thiết phải gắn liền với sự tồn tại của lớp xoáy ở phía hạ lưu của mép sau, theo một xấp xỉ đầu tiên, song song với dòng chảy ngược dòng. Do đó, chúng tôi được dẫn đến sơ đồ hóa dòng chảy bằng một lớp xoáy bao gồm đường bóng liên kết với cánh, đường bóng đánh thức tự do và dòng xoáy bắt đầu. Theo quy định Helmholtz, bề mặt không liên tục này bao gồm một phân phối liên tục các sợi xoáy dạng vòng khép kín (hình 10.4). Kết quả xoáy cuối cuốn mép tấm song song với dòng chảy ngược dòng, dưới tác dụng của vận tốc do toàn bộ tầng chứa nước gây ra. Chữ "U" được hình thành bởi các xoáy cuối và xoáy đầu khởi động không ở trong mặt phẳng của cánh. Các vận tốc cảm ứng dẫn đến một chuyển động đi xuống của toàn bộ phần đuôi (Hình 10.4).

\section{Hệ thống xoáy}

        Chúng ta mô hình hóa lực nâng của cánh bởi một hệ các xoáy xung quanh biên dạng cánh. Xoáy này được chia thành ba thành phần : xoáy khởi động, xoáy cạnh đuôi và xoáy biên. Chúng ta sẽ xử lý từng thành phần một.

\subsection{Xoáy khởi động}

        Khi một cánh được gia tốc từ trạng thái nghỉ ngơi, chuyển động xung quanh nó, và do đó, thang máy, không được sản xuất ngay lập tức. Thay vào đó, tại thời điểm bắt đầu, sắp xếp hợp lý trên phần sau của phần cánh như trong Hình. 7.1, với một điểm đình trệ xảy ra ở bề mặt trên phía sau. Tại cạnh sắc nét, không khí được yêu cầu để đổi hướng đột ngột khi vẫn đang chuyển động với tốc độ cao. Cuộc gọi tốc độ cao này đối với gia tốc cục bộ cực cao tạo ra lực nhớt rất lớn, và không khí không thể quay quanh mép sau đến điểm đình trệ. Thay vào đó, luồng không khí rời khỏi bề mặt và tạo ra một dòng xoáy ngay phía trên mép sau. Trà (mặt trên) điểm đình trệ di chuyển về phía mép sau, khi lưu thông xung quanh cánh và do đó lực nâng của nó tăng lên. Khi điểm đình trệ đạt đến mép sau, không khí không còn phải chảy xung quanh mép sau nữa. Thay vào đó dần dần giảm tốc dọc theo bề mặt cánh gió, dừng lại ở mép sau, và sau đó tăng tốc từ phần còn lại theo một hướng khác (Hình 7.2). Dòng xoáy bị bỏ lại phía sau tại điểm mà cánh đạt tới khi điểm đình trệ chạm tới mép sau. Phản ứng của nó, sự lưu thông quanh cánh, được ổn định ở giá trị cần thiết để7.1 Hệ thống xoáy 451 chuyến bay. đặt điểm đình trệ ở mép sau (xem Phần 6.1.1). phía sau có sức mạnh bằng nhau và ngược lại với ý nghĩa của sự lưu thông xung quanh cánh và được gọi là dòng xoáy khởi đầu hay dòng xoáy ban đầu.









Khi một cánh được gia tốc từ trạng thái nghỉ, lưu số không được tạo thành ngay lập tức. Thay vào đó, tại thời điểm bắt đầu, đường dòng có một điểm dừng xảy ra ở mặt trên phía sau. Tại cạnh đuôi, không khí phải đổi hướng đột ngột khi vẫn đang chuyển động với tốc độ cao và sự thay đổi này tạo ra lực nhớt rất lớn, và không khí không thể xoay quanh mép sau đến điểm dừng.

        Thay vào đó, luồng không khí rời khỏi bề mặt và tạo ra một dòng xoáy ngay phía trên mép sau. Điểm dừng di chuyển về phía mép sau, khi lưu thông xung quanh cánh tăng lên. Khi điểm dừng đạt đến cạnh sau, không khí không còn phải chảy xung quanh mép sau nữa. Thay vào đó nó giảm dần tốc độ dọc theo bề mặt biên dạng, dừng lại ở mép sau, và sau đó  tăng tốc từ phần còn lại theo một hướng khác. Dòng xoáy bị bỏ lại phía sau tại điểm mà cánh đạt tới khi điểm đình trệ chạm tới mép sau.


\end{document}