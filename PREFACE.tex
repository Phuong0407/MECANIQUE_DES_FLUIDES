\documentclass[main.tex]{subfiles}
\begin{document}
	Cơ học lưu chất là một ngành khoa học đã phát triển từ rất lâu đời, có lẽ là đã từ thời chiếc máy bơm nước trục vít của Archimède. Từ đó đến nay, trải qua bao thăng trầm cùng lịch sử nhân loại, cơ học lưu chất đã đạt được nhiều thành tựu trong nhiều phương diện, ảnh hưởng trực tiếp đến mọi ngóc ngách trong đời sống thường nhật con người. Do đó, việc có được những tri thức cơ bản nhất của cơ học lưu chất là một điều nên thực hiện. Đó là lý do cơ bản nhất để thôi thúc chúng tôi viết ra tác phẩm này.
	
	Tuy nhiên, nhóm tác giả cũng rất dè dặt khi biên soạn tác phẩm này, bởi vì chúng tôi không muốn sáng tạo lại cái bánh xe mà thay vào đó là tổ chức lại nó một cách hợp lý hơn để việc tiếp cận cơ học lưu chất trở nên dể dàng hơn.
	
	Do đó, trong tác phẩm này, quý vị sẽ thấy chúng tôi trình bày lại gần như là mọi thứ đã được thực hiện trong hàng thế kỷ qua về cơ học lưu chất để có cái .
	
	Về phần mình, chúng tôi gửi lời chân thành cảm ơn đến các thầy cô bộ môn Kỹ thuật Giao thông, đặc biệt xin cảm ơn thầy Nguyễn Thiện Tống, cô Lê Thị Hồng Hiếu, cô Nguyễn Song Thanh Thảo và thầy Đặng Trung Duẩn vì những giờ học bổ ích ở trên giảng đường cùng với những kiến thức khoa học chính xác. Bên cạnh đó, chúng tôi cũng xin cảm ơn những lời trao đổi khoa học hết sức chân thành và thẳng thắn đến từ các bạn sinh viên lớp VP19HK, đặc biệt cảm ơn các bạn Bùi Gia Bảo, Nguyễn Quốc Mạnh, Lê Trọng Đạt, cùng với các bạn Phạm Lê Tâm vì những đóng góp chân thành nhất về cách tiếp cận cũng như là những khúc mắc mà chúng ta gặp phải trong quá trình làm việc với nhau và với cơ học lưu chất. Chúng tôi cũng xin chân thàng cảm ơn các tác giả của hệ thống gõ văn bản \LaTeX, các phầm mềm soạn thảo TeXMaker, TexStudio, VSCode, LyX, cũng như các phần mềm biên tập đồ thị GnuPlot, và Git vì nếu không có các bộ công cụ này, không biết đến bao giờ chúng tôi mới có thể hoàn thành được tác phẩm này.
\end{document}