\documentclass[main.tex]{subfiles}
\begin{document}
Cơ học lưu chất là một ngành khoa học đã phát triển từ rất lâu đời, có lẽ là đã từ thời chiếc máy bơm nước trục vít của \textsc{Archimède}. Từ đó đến nay, cơ học lưu chất đã đạt được nhiều thành tựu và ảnh hưởng trực tiếp đến mọi ngóc ngách trong đời sống của con người. Do đó, việc có được những tri thức cơ bản nhất của cơ học lưu chất là một điều cần thiết, đặc biệt là đối với các khoa học gia và kỹ sư. Đó là lý do cơ bản nhất để chúng tôi viết ra tác phẩm này.

Khi viết quyển sách này, chúng tôi không dám đặt mình ngang hàng với các bậc tiền nhân mà chỉ muốn hệ thống hóa lại các mảng kiến thức rời rạc của cơ học lưu chất được giảng dạy tại các trường đại học ở Việt Nam, kết hợp với phương pháp tiếp cận tiên đề của mình để tạo nên một quyển sách với tri thức khoa học chính xác, phong phú và dể tiếp cận. Chúng tôi không sáng tạo lại cái bánh xe ! Chúng tôi chỉ sắp xếp lại nó để sự vận hành được hiệu quả hơn. Với ý niệm đó, chúng tôi hy vọng quyển sách này sẽ là một người bạn, tuy thụ động nhưng luôn luôn có mặt và hơn hết là đáng tin cậy, dành cho các bạn sinh viên cũng như những ai yêu thích cơ học lưu chất.

Như đã đề cập, chúng tôi sẽ sử dụng phương pháp tiên đề và xây dựng một nền tảng vững chắc hơn cho môn cơ học lưu chất thay vì các phương pháp tiếp cận “làm nhiều quen tay” như thông thường. Mọi phát biểu đều được chúng tôi chứng minh một cách đầy đủ và tường tận, hoặc trong trường hợp không chứng minh được, chúng tôi sẽ nêu ra nguồn gốc của chúng. Ở những chổ quan trọng, tùy tình huống, sẽ có những đoạn bình luận hoặc chú ý hoặc nhận xét một cách đầy đủ. Các đoạn văn hỗ trợ này là sự quan sát nhiều lần những sự ngộ nhận của các sinh viên trong việc học và thực hành cơ học lưu chất, do đó hy vọng chúng sẽ mang những giá trị hữu ích cho quý độc giả. Bên cạnh đó, chúng tôi không bao giờ coi thường lịch sử, do đó ở những chổ cần thiết, các đoạn văn bản miêu tả tiến trình lịch sử liên quan đến sự phát triển của cơ học lưu chất đều sẽ được nêu ra.

Không giấu gì quý độc giả, nhóm tác giả có sự khao khác đặc biệt với cái đẹp, và là những người theo chủ nghĩa hoàn hảo. Do đó, chúng tôi sẽ cố gắng biến một tác phẩm khoa học như thế này thành một tác phẩm nghệ thuật với các kiểu chữ được trình bày trau chuốt đi kèm với một bố cục rõ ràng và một lược đồ sáng sủa cho mọi vấn đề được đề cập. Hy vọng với hình thức này, chúng tôi sẽ kích thích được sự đam mê khoa học của quý độc giả. Chúng tôi cũng xin được xưng hô \bi{chúng ta}, hoặc đôi khi là \bi{ta}, để tránh đi một sự “khô cứng” trong xưng hô và tạo sự gần gủi hơn nữa. 

Để hiện thực hóa những điều vừa kể, chúng tôi sẽ tổ chức tác phẩm như sau. Đầu tiên chúng tôi sẽ nêu bật lên khái niệm về các lưu chất cũng như phương pháp tiếp cận được sử dụng để mô hình hóa các lưu chất, ở đây, chúng tôi sẽ cố gắng sử dụng cách tiếp cận \bi{hiện tượng học} thay vì một cách tiếp cận vi mô khó nắm bắt hơn (tuy nhiên sẽ chính xác hơn). Cách tiếp cận này là những gì đã được thực hiện ở cơ học môi trường liên tục và chúng ta sẽ nhấn mạnh hơn về sự hợp lệ của sự mô hình hóa liên tục. Sau đó hai phương pháp mô tả chuyển động kinh điển của lưu chất sẽ được giới thiệu, phương pháp Lagrange và phương pháp Euler. Ở đây, chúng tôi sẽ nhấn mạnh về sự cần thiết của cả hai phương pháp này, và không đánh giá quá cao một phương pháp nào vì chúng đều có ưu nhược điểm của riêng chúng. Lưu chất thông thường là một môi trường liên tục, do đó không bao giờ có thể chối bỏ sự biến dạng của nó. Như vậy, chúng ta sẽ đề cập đến sự biến dạng của lưu chất theo cách tiếp cận môi trường liên tục. Tuy nhiên lưu chất dể bị biến dạng và dường như là khó đạt được trạng thái tĩnh, do đó, sẽ thật tự nhiên nếu chúng ta, trong phần lớn thời gian, chỉ đề cập đến biến đổi của biến dạng theo thời gian.

Về phần mình, chúng tôi gửi lời chân thành cảm ơn đến các thầy cô bộ môn Kỹ thuật Giao thông, đặc biệt xin cảm ơn thầy Nguyễn Thiện Tống, cô Lê Thị Hồng Hiếu, cô Nguyễn Song Thanh Thảo và thầy Đặng Trung Duẩn vì những giờ học bổ ích ở trên giảng đường cùng với những kiến thức khoa học chính xác. Bên cạnh đó, chúng tôi cũng xin cảm ơn những lời trao đổi khoa học hết sức chân thành và thẳng thắn đến từ các bạn sinh viên lớp VP19HK, đặc biệt cảm ơn các bạn Bùi Gia Bảo, Nguyễn Quốc Mạnh, Lê Trọng Đạt, cùng với các bạn Phạm Lê Tâm, Nguyễn Phạm Cao Quân vì những đóng góp chân thành nhất về cách tiếp cận cũng như là những khúc mắc mà chúng ta gặp phải trong quá trình làm việc với nhau và với cơ học lưu chất. Chúng tôi cũng xin chân thàng cảm ơn các tác giả của \LaTeX, các phầm mềm soạn thảo TeXMaker, TeXStudio, VSCode, LyX, các phần mềm biên tập đồ thị GnuPlot, LatexDraw, và công cụ quản lý phiên bản Git vì nếu không có các bộ công cụ này, không biết đến bao giờ chúng tôi mới có thể hoàn thành được tác phẩm này.

\end{document}