\documentclass[THUY_LUC_KHI_NEN.tex]{subfiles}

\begin{document}
\chapter{HỆ THỐNG TRUYỀN ĐỘNG THỦY LỰC VÀ ĐIỀU KHIỂN CHÚNG}
\section{Giới thiệu}
Thủy lực là một trong những hệ thống truyền động công suất lớn lâu đời đời nhất, mặc dù sự phát triển của hệ thống 

Độ tin cậy và tuổi thọ tăng lên khi sử dụng chất lỏng gốc dầu và các bộ phận làm kín bằng cao su nitrile đã tạo ra một bước phát triển trong việc sử dụng các hệ thống truyền động công suất lớn dựa trên lưu chất áp dụng cho máy. Một số ưu điểm mà năng lượng thủy lực có so với các hệ thống truyền động công suất khác là :
\begin{itemize}
    \item Hệ thống không bị giới hạn về kích thước ràng buộc bởi các bánh răng và trục truyền động thông thường.
    \item Tốc độ có thể được kiểm soát vô cấp với mức tăng tương đối ít
    trong độ phức tạp của mạch.
    \item Tỷ lệ công suất trên khối lượng cao cho phép phản hồi nhanh và trọng lượng lắp đặt là nhỏ.
    \item Xuất lực có được không phụ thuộc vào tốc độ vận hành. Tải trọng  có thể được duy trì trong thời gian không xác định.
\end{itemize}

Thiết bị điều khiển điện tử đã được ứng dụng rộng rãi trong máy móc, đặc biệt khi cần vận hành bởi máy tính hoặc bộ điều khiển logic khả trình (PLC). Các thiết bị điện tử đã cải thiện độ chính xác của việc điều khiển bằng cách sử dụng kỹ thuật điều khiển vòng lặp trong nhiều ứng dụng được vận hành một cách truyền thống bởi các hệ thống vòng lặp thủy lực.

Việc truyền tải công suất bằng chất lỏng có được chấp nhận thể hay không  phụ thuộc vào yêu cầu của hệ thống.
\section{Thiết kế hệ thống truyền động bằng lưu chất}
Có nhiều loại hệ thống truyền động bằng chất lỏng được sử dụng. Loại mạch được sử dụng phụ thuộc vào các linh kiện của hệ thống và sự lựa chọn của người dùng và do đó, điều này thường có ảnh hưởng quan trọng đến các linh kiện của hệ thống. Tuy nhiên, có kỹ thuật để đánh giá hiệu suất của hệ thống, nhà thiết kế cần phải biết để lựa chọn cả loại mạch được sử dụng và các linh kiện của chúng.
\subsection{Sự lựa chọn các linh kiện}
Các mạch có thể được sắp xếp theo nhiều cách khác nhau. Ngoài ra, có nhiều loại linh kiện khác nhau có sẵn để thực hiện một chức năng cụ thể và vì
vậy, quá trình lựa chọn linh kiện không dễ dàng vì nó yêu cầu kiến thức về:
\begin{itemize}
    \item Sự có sẳn của các linh kiện thủy lực.
    \item Đặc tính hoạt động của các linh kiện và việc sử dụng chúng trong mạch và hệ thống điều khiển.
    \item Các mạch thủy lực có sẳn.
    \item Phương pháp phân tích để xác định hiệu suất hệ thống để đáp ứng các yêu cầu đề ra.
\end{itemize}
\subsection{Sự lựa chọn mạch}
Các yếu tố quyết định để lựa chọn mạch thủy lực :
\begin{itemize}
    \item Chi phí ban đầu.
    \item Khối lượng.
    \item Sự dể dàng bảo trì.
    \item Chi phí vận hành.
    \item Vòng đời thiết bị.
\end{itemize}
\subsection{Quá trình thiết kết hệ thống}
\begin{itemize}
    \item Tính toán các đặc tính của linh kiện và xác định loại của hệ thống được sử dụng.
    \item Thiết lập kiểu và kích cở của các linh kiện thủy lực quan trọng.
    \item Chọn thiết kế cho phù hợp với mạch thủy lực.
    \item Thực hiện các phép phân tích hệ thống và xác định khả năng đạt được nhu cầu.
\end{itemize}
\end{document}