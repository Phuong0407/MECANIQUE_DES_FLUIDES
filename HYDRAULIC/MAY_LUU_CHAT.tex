\documentclass[13.5pt,twoside,a4paper]{extbook}
\usepackage[utf8]{vietnam}
\usepackage[inner=0.5cm,outer=2cm,top=2cm,bottom=2cm,bindingoffset=1.5cm]{geometry}
\usepackage{subfiles}
\usepackage{indentfirst}

\begin{document}
\chapter{MÁY LƯU CHẤT}
\section{Giới thiệu và phân loại}
Máy lưu chất được phân chia tự nhiên thành loại cấp năng lượng cho lưu chất (\emph{máy bơm}) và loại lấy năng lượng từ lưu chất (\emph{tuabin}). Máy vận chuyển chất lỏng được gọi đơn giản là máy bơm, nhưng nếu có liên quan đến khí, ba thuật ngữ khác nhau đang được sử dụng, tùy thuộc vào mức tăng áp suất đạt được. Nếu áp suất tăng rất nhỏ (một vài inch nước), máy bơm khí được gọi là \emph{quạt}; lên đến 1 atm, nó thường được gọi là \emph{máy thổi}; và trên 1 atm nó thường được gọi là \emph{máy nén}.
\subsection{Phân loại máy bơm}
Có hai loại máy bơm cơ bản : máy bơm dịch chuyển tích cực và máy bơm động lực.

Máy bơm dịch chuyển tích cực (PDP) đẩy chất lỏng cùng với sự thay đổi thể tích. Lỗ mở ra và chất lỏng được đưa vào qua một cửa vào. Sau đó, khoang đóng lại, và chất lỏng được ép qua một cửa ra. Tất cả các PDP đều cung cấp một dòng chảy hoặc xung động hoặc tuần hoàn khi thể tích khoang mở ra, bẫy và ép lưu chất. Lợi thế lớn của chúng là có thể phân phối bất kỳ chất lỏng với độ nhớt bất kỳ. Vì PDP nén một cách cơ học trong một khoang chứa đầy chất lỏng, chúng áp đặt một áp lực lớn nếu cổng ra bị tắt. Cần phải được xây dựng chắc chắn và việc ngắt hoàn toàn sẽ gây ra thiệt hại nếu van giảm áp không được sử dụng.

Máy bơm động lực thêm động lượng cho chất lỏng bằng cách di chuyển nhanh
lưỡi hoặc cánh gạt hoặc một số thiết kế đặc biệt. Không có thể tích đóng : Chất lỏng tăng động lượng khi di chuyển qua các đoạn mở và sau đó chuyển đổi tốc độ cao thành áp suất bằng cách thóat ra và đi vào một bộ khuếch tán. Máy bơm động lực thường cung cấp dòng chảy với tốc độ cao hơn PDP và xả ổn định hơn nhiều nhưng không hiệu quả trong việc xử lý chất lỏng có độ nhớt cao. Máy bơm động lực nói chung cũng cần sơn lót; nếu chúng chứa đầy khí, chúng không thể hút chất lỏng
từ bên dưới vào đầu vào của họ. Mặt khác, PDP có khả năng tự mồi cho hầu hết các ứng dụng. Một máy bơm động lực có thể cung cấp tốc độ dòng chảy rất cao (lên đến 300.000 gal / phút)
nhưng thường với áp suất tăng vừa phải (một vài atm). Ngược lại, PDP có thể


\end{document}