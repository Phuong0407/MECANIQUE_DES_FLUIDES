\documentclass[THUY_DONG_HOC.tex]{subfiles}

\begin{document}
\chapter{SỰ KHUẾCH TÁN ĐỘNG LƯỢNG VÀ CHẾ ĐỘ DÒNG CHẢY}
\section{Vận chuyển khuéch tán và đối lưu động lượng trong dòng chảy}
\subsection{Sự khuếch tán và sự đối lưu động lượng : hai thí nghiệm}
Sự vận chuyển động lượng qua sự đối lưu có thể được hiểu đơn giản thông qua ví dụ về một chất lỏng chảy song song với vecteur vận tốc $\textbf{U}$ hằng số. Mỗi phần tử của lưu chất được vận chuyển động lượng trong khi chúng dịch chuyển với vận tốc riêng, là vận tốc cục bộ $\textbf{U}$ của dòng chảy. Thông lượng của động lượng trong một đơn vị thời gian và một đơn vị diện tích của một ống bằng với tích $\textbf{U}$ và một đại lượng vận chuyển $\rho\textbf{U}$. Số hạng $\rho U^2$ có thứ nguyên là áp suất, nó nhân với $1/2$ chính là áp suất của lưu chất;

Chúng ta sẽ thấy rằng sự vận chuyển động lượng thông qua sự khuếch tán cũng là một cơ chế hiệu quả, nhưng nó thường bị che khuất khi vận chuyển bằng đối lưu. Bởi vì sự vận chuyển đối lưu động lượng thường được diển ra theo hướng dòng chảy, người ta sẽ dễ dàng xác định sự khuếch tán theo hướng vuông góc hơn, khi chúng ta quan sát nó trong thí nghiệm được mô tả bởi Hình 2.1. Một hình trụ có trục hướng thẳng đứng và bán kính $R$ chứa đầy một chất lỏng mà chuyển động của nó có thể được hình dung bởi các hạt lắng đọng trên bề mặt của nó. Hệ ban đầu ở trạng thái nghỉ. Tại thời điểm ban đầu, chúng ta đặt một chuyển động xoay với vận tốc góc cố định $\Omega_0$. Trước hết, chỉ các lớp chất lỏng liền kề với thành hình trụ chuyển động với vận tốc góc của hình trụ (Hình 2.1a). Dòng chảy lưu chất được đặc trưng bởi vận tốc góc $\Omega(r,t)=v(r,t)/r$, trong đó vận tốc địa phương $v$ của dòng chảy được định hướng theo phương trực xuyên tâm. Dòng chảy truyền theo từng lớp từ ngoài vào trong, theo thời gian dài, chất lỏng quay với vận tốc góc đều bằng vận tốc của hình trụ (Hình 2.1b). Hiện tượng này rất giống với vấn đề khuếch tán nhiệt mà chúng ta đã thảo luận trong chương 1 : chúng ta coi một hình trụ rắn được tạo thành từ vật liệu khuếch tán nhiệt $\kappa$ ở nhiệt độ đồng đều $T_0$; tại thời điểm ban đầu, thành bên ngoài của hình trụ này được đưa đến nhiệt độ $T_0 + \Delta T_0$. Sự nhiễu loạn nhiệt độ lan truyền thông qua khuếch tán vào các lớp bên trong, và độ dày của vùng bị ảnh hưởng tăng lên theo thời gian như $\sqrt{\kappa t}$; cùng quy luật lan truyền theo $\sqrt{t}$ được quan sát trong thí nghiệm thủy động lực học. Hơn nữa, chúng ta sẽ thấy rằng, trong thí nghiệm dòng chảy trong một hình trụ quay, chúng ta cũng có thể xác định một hệ số khuếch tán của động lượng : điều này làm cho nó tương thích chặt chẽ với profil vận tốc góc $\Omega(t)$ tại các thời điểm khác nhau, với profil khuếch tán nhiệt $\delta T(r)$ của Hình 1.9.

Chúng ta thấy có sự vận chuyển "động lượng" từ lớp này sang lớp kia thông qua sự đối lưu theo phương bán kính; sự đối lưu do dòng chảy thủy động học thực sự không thể đóng góp cho sự lan truyền này bởi vì chất lưu chuyển động theo phương trực xuyên tâm. Điểm quan trọng thứ hai của thí nghiệm này là vận tốc của thành rắn bằng với vận tốc lưu chất ở gần thành. Tính chất này được quan sát thấy đối vơis mọi lưu chất nhớt thông thường. Tồn tại lực ma sát giữa lưu chất và thành bình. Sự vận chuyển khuếch tán động lượng được đảm bảo bởi độ nhớt, là một đặc tính của lưu chất mà bây giờ chúng ta sẽ thảo luận từ quan điểm vĩ mô.
\begin{description}
		\item[Chú ý :] Mô tả mà chúng tôi đã trình bày có giá trị đối với một hình trụ dài vô hạn. Trong thực tế, đáy của hình trụ đóng một vai trò quan trọng trong sự biến đổi vận tốc khi thời gian tăng lên, do nó tạo ra dòng chảy thứ cấp, như chúng ta sẽ thấy trong chương 7 (Phần 7.6).
\end{description}
\subsection{Sự vận chuyển động lượng trong dòng chảy phân tầng - giới thiệu về độ nhớt}
\noindent{\textbf{Định nghĩa vĩ mô về độ nhớt}}\\

Chúng ta xét bài toán không tĩnh, trong đó vận tốc phụ thuộc tường minh vào thời gian. Dòng chảy của chúng ta được đặt giữa hai mặt phẳng vô hạn song song với nhau và có khoảng cách $a$ theo hướng $y$. Mặt phẳng được đặt tại $y=0$ là đứng yên vằ mặt phẳng còn lại được đặt ở $y=a$ di chuyển theo hướng $Ox$ với vận tốc $V_0$. Lưu chất có profile vận tốc như sau :
\begin{equation}
	\begin{aligned}
		v_x(y)=V_0\frac{y}{a}.
	\end{aligned}
\end{equation}
Dòng chảy này được gọi là \emph{dòng chảy Couette phẳng}. 







\end{document}