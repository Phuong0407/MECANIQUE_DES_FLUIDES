\documentclass[THUY_DONG_HOC.tex]{subfiles}

\begin{document}
\chapter{ĐỘNG LỰC HỌC LƯU CHẤT NHỚT}

\newpage

\section{Mô hình hóa lực trong một lưu chất nhớt}

Như đã thảo luận ở phần lưu chất lý tưởng, bây giờ chúng ta sẽ tiến hành mô hình hóa thành phần tiếp tuyến $\underline{F}_T$ của lực bề mặt. Lực này phản ánh ma sát giữa các lớp chất lỏng trượt tương đối với nhau, và là do độ nhớt của chất lỏng. Thật vậy, độ nhớt là một hệ số vận chuyển phản ánh sự truyền động lượng từ các miền có vận tốc cao hơn đến các miền có vận tốc thấp hơn.

Tuy nhiên, lực bề mặt khó có thể được tính đến trong phương trình chuyển động, do đó chúng ta sẽ tính đương lượng thể tích của nó, để làm được điều này, chúng ta sẽ định nghĩa mật độ bề mặt của lực bề mặt này :
\begin{equation}
	\begin{aligned}
		\underline{F}(M,t)=\underline{\mathcal{F}}(M,t)dS;
	\end{aligned}
\end{equation}
Sử dụng định lý Gauss-Odstrogradsky cho phép ta viết :
\begin{equation}
	\begin{aligned}
		F(M,t)=\underline{\mathcal{F}}(M,t)\underline{n}dS;
	\end{aligned}
\end{equation}


Trên khối này, tổng lực bề mặt tác dụng lên khối lưu chất này được tính :
\begin{equation}
	\begin{aligned}
		\underline{F}(M,t)=\varoiint_\mathscr{S} \mathcal{F}(M,t)\underline{n}dS=\iiint_\mathscr{V} \underline{}\underline{\underline{\sigma}}
	\end{aligned}
\end{equation}

Từ đây, chúng ta dẫn đến một khái niệm tenseur ứng suất :
$$
\sigma_{ij}=-p\delta_{ij}+\sigma'_{ij}.
$$

Tuy nhiên



\end{document}