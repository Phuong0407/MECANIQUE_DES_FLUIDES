\documentclass[CO_LUU_CHAT_NC.tex]{subfiles}
\begin{document}
\chapter{GIỚI THIỆU VỀ LÝ THUYẾT ỔN ĐỊNH DÒNG CHẢY VÀ SỰ BẤT ỔN ĐỊNH}
\section{Sự phân đôi}
Xét một dòng chảy của một lưu chất nhớt không nén được trong một miền cho trước $\mathcal{V}$. Phương trình Navier-Stokes được viết :
\begin{equation}
    \begin{aligned}
        \frac{\partial\underline{u}}{\partial t}+\left(\underline{u}\cdot\underline{\nabla}\right)\underline{u}=-\frac{1}{\rho}\underline{\nabla}p+\nu\Delta\underline{u}.
    \end{aligned}
\end{equation}
Phương trình bảo toàn khối lượng :
\begin{equation}
    \begin{aligned}
        \underline{\nabla}\cdot\underline{u}=0.
    \end{aligned}
\end{equation}
Điều kiện biên được viết :
\begin{equation}
    \begin{aligned}
       \underline{u}=\underline{U}.
    \end{aligned}
\end{equation}
trong đó $\underline{U}$ là vận tốc của lưu chất trên bề mặt. Để nghiên cứu phương trình Navier-Stokes, chúng ta đơn giản hóa nó thành phương trình vô thứ nguyên :
\begin{equation}
    \begin{aligned}
        \frac{\partial\underline{u}}{\partial t}+\left(\underline{u}\cdot\underline{\nabla}\right)\underline{u}=-\frac{1}{\rho}\underline{\nabla}p+\nu\Delta\underline{u}.
    \end{aligned}
\end{equation}










\end{document}