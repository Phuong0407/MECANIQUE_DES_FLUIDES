\documentclass[../../../main.tex]{subfiles}

\begin{document}
    Cơ học lưu chất có một ứng dụng rất rộng rãi trong rất nhiều hệ khác nhau như dòng khí chuyển động qua cánh máy bay, máu chảy trong mạch máu, hoặc là dòng dầu thô chảy trong các đường ống dài hàng ngàn km. Trong tác phẩm này, chúng ta sẽ đi vào nghiên cứu một loạt các hệ thống lưu chất khác nhau.
        
    Trước tiên chúng ta nên thảo luận về hệ thống triết học nền tảng nhất của môn cơ học lưu chất. Như tên gọi của nó, môn học \bi{Cơ học lưu chất} là một môn học nghiên cứu về các ứng xử cơ học của các lưu chất. Lưu chất là một khái niệm để gọi chung cho chất lỏng và chất khí khi ta quan sát hệ lưu chất ở thang đo vĩ mô và thang đo trung mô\footnote{Ta sẽ không đưa vào đây các nghiên cứu về plasma.}.

    Thang đo vĩ mô ở đây là thang đo mà chúng ta vẫn làm việc hằng ngày. Kích cở đặc trưng của thang đo này là phụ thuộc vào kích thước của hệ được nghiên cứu, do đó ta sẽ nhấn mạnh ở đây lối nghiên cứu điển hình (études des cas). Các nghiên cứu điển hình sẽ được thực hiện song song với các nghiên cứu phổ quát, trong đó tùy từng đối tượng cụ thể mà chúng tôi sẽ giới thiệu các cách tiếp cận cùng với các tham số liên quan.

    Ở thang đo vi mô, chất lỏng và chất khí là khác nhau một cách nền tảng. Đầu tiên mật độ phân tử của các phân tử không khí là vào cở $\SI{e25}{\per\metre\cubed}$, còn mật độ phân tử của nước chẳng hạn là vào cở $\SI{e28}{\per\metre\cubed}$. Mật độ phân tử của chất lỏng do đó là lớn hơn rất nhiều so với chất khí, do đó lực hút giữa các phân tử chất lỏng là đáng kể. Tập tính của chất lỏng và chất khí do đó là khác nhau, chẳng hạn như khả năng tẩm ướt và khả năng hòa tan của chất lỏng; còn chất khí thì dể dàng bị nén hơn. Một điều cần phải nhấn mạnh là ở thang đo vi mô, lưu chất không được xem là các khối liên tục nữa và sự xử lý lúc này phải được thực hiện bằng các công cụ của cơ học thống kê.

    Hiển nhiên, trong phần lớn các nghiên cứu của chúng ta, ta vẫn sẽ đứng trong cơ sở sự mô hình hóa \bi{cơ học môi trường liên tục}\footnote{Một sự nghiên cứu đầy đủ cơ học môi trường liên tục là cần thiết trước khi bước vào đọc quyển sách này. Tuy nhiên, để tránh làm thất vọng quý độc giả, chúng tôi sẽ trình bày đầy đủ các kiến thức nền tảng của cơ học môi trường liên tục rồi mới chính thức đi vào cơ học lưu chất.} và đưa vào thang đo trung mô. Với thang đo mới này, lưu chất được chia thành các khối có thể tích đủ nhỏ ở thang đo vĩ mô để xem các đại lượng cơ học liên kết với khối này là không đổi nhưng nó vẫn đủ lớn ở thang đo vi mô để xem khối này là liên tục.
        
    Nếu gọi kích thước đặc trưng của hệ cơ học môi trường ở thang đo vĩ mô là $L$\footnote{Với một ống hình trụ, có thể chọn kích cở đặc trưng là đường kính ống.}, kích thước đặc trưng của môi trường vi mô là $l$ (thường chọn quảng đường tự do trung bình của phân tử lưu chất). Kích thước đặc trưng của một khối trung mô được kí hiệu là $\delta$ phải thoả mãn:
        \begin{align}
            l\ll\delta\ll L,
        \end{align}
    để sự mô hình hóa môi trường liên tục là khả dĩ.

    Sự mô tả môi trường liên tục là tốt trong gần như mọi trường hợp, nhưng những sự không liên tục vẫn có thể diễn ra ở mức độ trung mô, ta đang đề cập đến trường hợp các \bi{sóng xung kích} đối với các máy bay trên âm hoặc \bi{sự xâm thực} đối với các chong chóng tàu thủy. Ta cũng sẽ tính đến các hiện tượng này trong sự mô tả môi trường liên tục và bây giờ ta sẽ mở rộng các khảo sát ra các đại lượng có sự liên tục từng khúc.
\end{document}