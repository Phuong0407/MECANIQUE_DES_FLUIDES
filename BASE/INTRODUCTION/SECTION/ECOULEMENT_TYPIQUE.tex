\documentclass[../../../main.tex]{subfiles}

\begin{document}
	Sau khi đã nghiên cứu sự biến dạng của lưu chất, bây giờ ta sẽ tiến hành mô tả một vài loại dòng chảy điển hình. Mục tiêu của chúng ta là tìm ra một vài khuôn mẫu dòng chảy đặc biệt, để có thể sử dụng về sau trong các nghiên cứu điển hình về sau.
	\subsubsection{Dòng chảy không nén được}
		Một dòng chảy phẳng không nén được là một dòng chảy có $\underline{\nabla}\cdot\underline{u}=0$. Điều này chứng tỏ rằng tồn tại một trường vecteur $\underline{A}$ sao cho:
			\begin{align}
				\underline{u}\left(\underline{x},t\right)=\underline{\nabla}\wedge\underline{\Psi}\left(\underline{x},t\right).
			\end{align}
		Ta vừa mới chuyển từ một trường vecteur sang một trường vecteur khác. Nói một cách trực diện hơn, ta đã chuyển từ một sự phức tạp sang một sự phức tạp khác. Do đó, công thức này chỉ có ý nghĩa về mặt lý thuyết.

		Nếu dòng chảy này nằm trong mặt phẳng có pháp tuyến là $\underline{e}_z$ thì tồn tại một hàm vecteur $\underline{\Psi}=\Psi\underline{e}_z$ sao cho:
			
		Bằng các công thức đã thực hiện trong giải tích vecteur, ta có:
			\begin{align}
				\underline{u}\left(\underline{x},t\right)=\underline{\nabla\Psi}\left(\underline{x},t\right)\wedge\underline{e}_z.
			\end{align}
		Do đó ta gọi hàm số $\Psi$ là \bi{hàm dòng}. Ta sẽ trở lại các nghiên cứu sâu hơn về dòng chảy này về sau.
	\subsubsection{Dòng chảy xoay}
		
\end{document}