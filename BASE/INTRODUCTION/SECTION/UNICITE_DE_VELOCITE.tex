\documentclass[../../../main.tex]{subfiles}

\begin{document}
    Đối với một lưu chất đang chuyển động, sự khảo sát vận tốc của dòng chảy trong cùng một hệ quy chiếu $\mathcal{R}$ chỉ cho ta một giá trị vận tốc dù cho có sử dụng hình thức luận Lagrange hay hình thức luận Euler. Do đó phải có sự đồng nhất giữa vận tốc trong hai hình thức này.
    
    Như vậy, nếu một hạt lưu chất ban đầu ở vị trí $\underline{X}$ đi qua điểm $\underline{x}$ ở thời điểm $t$, vận tốc đo được trong hình thức luận Euler tại điểm $\underline{x}$ sẽ chính là vận tốc của hạt lưu chất này trong hình thức luận Lagrange, tức là:
    	\begin{align}
    		\underline{u}\left(\underline{x},t\right)=\underline{U}\left(\underline{X},t\right)\quad\Longleftrightarrow \quad\underline{x}=\underline{\phi}\left(\underline{X},t\right)
    	\end{align}
    Mặc dù là cùng một vận tốc nhưng việc xử lý toán học các vận tốc này là khác nhau tùy theo hình thức luận.

    \begin{description}
        \item[Nhận xét:] Mặc dù có vẻ là hình thức luận Euler đơn giản hơn về sự mô tả, ta sẽ không từ bỏ hình thức luận Euler. Lý do cho việc này là do mỗi hình thức luận có các điểm mạnh và các điểm yếu của chúng. Đối với hình thức luận Lagrange, ta phải theo dõi từng hạt một (chắc chắn số lượng hạt là khổng lồ)và điều này là rất khó để thực hiện được về mặt thực nghiệm.
        \item[Nhận xét:] 
    \end{description}
\end{document}