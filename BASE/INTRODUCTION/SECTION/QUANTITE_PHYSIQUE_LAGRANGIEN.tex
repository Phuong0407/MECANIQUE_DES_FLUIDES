\documentclass[../../../main.tex]{subfiles}

\begin{document}
	Theo hình thức luận Lagrange, ta sẽ đi theo từng hạt lưu chất. Do đó khi muốn khảo sát một đại lượng nào đó, ta sẽ gắn nó với một hạt lưu chất và theo dõi sự biến thiên của nó. Cụ thể hơn, đối với một đại lượng vật lý $\mathscr{B}$, ta sẽ hình thức hóa nó bởi:
		\begin{align}
			\mathscr{B}=\mathscr{B}\left(\underline{X},t\right).
		\end{align}
	Như vậy, sự biến thiên của đại lượng này theo thời gian đơn giản chỉ là đạo hàm riêng theo thời gian:
		\begin{align}
			\dfrac{d\mathscr{B}}{dt}=\dfrac{\partial\mathscr{B}}{\partial t}\left(\underline{X},t\right).
		\end{align}
	Phép đạo hàm này được gọi là \bi{đạo hàm hạt}. Để phân biệt nó về mặt kí hiệu, ta viết:
		\begin{align}
			\boxed{\dfrac{D\mathscr{B}}{Dt}=\dfrac{\partial\mathscr{B}}{\partial t}\left(\underline{X},t\right)}.
		\end{align}
	
\end{document}