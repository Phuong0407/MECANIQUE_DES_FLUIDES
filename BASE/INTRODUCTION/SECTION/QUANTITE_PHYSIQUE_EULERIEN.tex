\documentclass[../../../main.tex]{subfiles}

\begin{document}
	Theo hình thức luận Euler, ta sẽ đứng yêu tại một điểm trong không gian và khảo sát một đại lượng nào đó. Do đó, ta chỉ còn có thể khảo sát sự biến thiên của đại lượng này theo thời gian và việc sử dụng khái niệm trường là không thể tránh khỏi. Cụ thể hơn, với một đại lượng vật lý $\mathscr{B}$, ta sẽ hình thức hóa nó bởi:
		\begin{align}
			\mathscr{B}=\mathscr{b}\left(\underline{x},t\right).
		\end{align}

	Sự biến thiên của đại lượng này theo thời gian không dể để khảo sát như lúc trước nữa vì giá trị của nó không còn là một giá trị nội tại của riêng từng lưu chất nữa. Bây giờ, giá trị đo được của nó tại một điểm cho trước là một giá trị mang tính tập hợp, có nghĩa là tùy theo hạt lưu chất đi qua điểm khảo sát.

	Tương tự như sự duy nhất của vận tốc, giá trị của đại lượng này chỉ có một bất chấp là hình thức luận nào đang được dùng để khảo sát. Do đó, giá trị đo được của đại lượng đó tại một thời điểm $t$ ở điểm $\underline{x}$ cố định chính là giá trị của đại lượng tương ứng của hạt lưu chất có mặt tại thời điểm đó, tức là:
		\begin{align}
			\mathscr{B}=\mathscr{b}\left(\underline{x},t\right)=\mathscr{B}\left(\underline{\phi}^{-1}\left(\underline{x},t\right),t\right).
		\end{align}
\end{document}