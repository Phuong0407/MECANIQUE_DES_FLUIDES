\documentclass[../../../main.tex]{subfiles}

\begin{document}
    Mặc dù có tên gọi là mô tả động học, ta sẽ đề cập không chỉ về động lực học của lưu chất mà còn về sự biến dạng của lưu chất. Như đã đề cập, lưu chất là môi trường liên tục, do đó, tính chất rất quan trọng của chúng là sự biến dạng\footnote{Không phải sự dời chổ, ta sẽ nói kỹ hơn về sự phân biệt về sự biến dạng và sự dời chổ.}. Ở đây ta sẽ không tiếp cận như cách mà cơ học môi trường liên tục, tức là không nghiên cứu \emph{độ dời} của các phần tử lưu chất, mà sẽ chỉ nghiên cứu thông qua vận tốc của các hạt lưu chất\footnote{Điều này là hiển nhiên, vì lưu chất rất linh động, chúng di chuyển dể dàng và không có hình dạng xác định do đó việc nghiên cứu lưu chất qua biến dạng là không hợp lý.}.
\subsection{Tốc độ biến dạng của một vecteur vật chất}
    Như đã đề cập, ta sẽ nghiên cứu sự biến dạng với trọng tâm là sự khảo sát trường vận tốc của lưu chất.

    Đầu tiên, ở một thời điểm xác định $t$, ta chọn bên trong lưu chất hai điểm vật chất\footnote{Tức là hai điểm nằm trên hai hạt lưu chất, ta muốn dùng từ này để nhấn mạnh lên tính chất của điểm này.}, $M_1$ và $M_2$, ở lân cận nhau. Ta kí hiệu vecteur nối hai điểm này là $\underline{dx}=\underline{M_1M_2}$. Để nghiên cứu chúng, ta sẽ truy ngược về hình thái $\kappa_0$, và ta sẽ tìm được hai vị trí $\underline{X}_1$ và $\underline{X}_2$ sao cho:
		\[
			\underline{OM_1}=\underline{\phi}\left(\underline{X}_1,t\right)\quad\text{và}\quad\underline{OM_2}=\underline{\phi}\left(\underline{X}_2,t\right).
		\]
	Ta kí hiệu vecteur vật chất trong hình thái $\kappa_0$ bởi $\underline{dX}=\underline{X_2}-\underline{X_1}=\underline{X_1X_2}$. Như vậy, vecteur vật chất này chịu sự xử lý toán học dưới đây:
		\[
			\begin{aligned}
				\underline{dx}&=\underline{\phi}\left(\underline{X}_2,t\right)-\underline{\phi}\left(\underline{X}_1,t\right)\\
				&=\dfrac{\partial\underline{\phi}}{\partial\underline{X}}\left(\underline{X}_1,t\right)\cdot\underline{dX}.
			\end{aligned}
		\]
	Do đó, khi đưa vào khái niệm \bi{gradient biến đổi} được định nghĩa như sau:
		\begin{align}
			\boxed{\underline{\underline{F}}\left(\underline{X},t\right)=\underline{\underline{\nabla_X\phi}}\left(\underline{X},t\right)}
		\end{align}
	(trong đó kí hiệu $\nabla_X$ muốn nhấn mạnh việc lấy gradient của một hàm vecteur trong hình thái tham chiếu), hệ thức trên được viết thành:
		\begin{align}
			\underline{dx}=\underline{\underline{F}}\left(\underline{X},t\right)\cdot\underline{dX}.
		\end{align}

    Ở một thời điểm $t+\delta t$ vô cùng bé sau đó, hai điểm này bị các hạt lưu chất tương ứng kéo đi và trở thành hai điểm vật chất mới mà ta kí hiệu tương ứng là $M'_1$ và $M'_2$. Do đó:
    	\[
			\underline{OM'_1}=\underline{\phi}\left(\underline{X}_1,t+\delta t\right)\quad\text{và}\quad\underline{OM'_2}=\underline{\phi}\left(\underline{X}_2,t+\delta t\right).
    	\]
    Vecteur vật chất mới này được xử lý toán học tương tự như ở phía trên:
		\[
			\begin{aligned}
				\underline{dx'}:=\underline{M'_1M'_2}&=\underline{\phi}\left(\underline{X}_2,t+\delta t\right)-\underline{\phi}\left(\underline{X}_1,t+\delta t\right)\\
				&=\dfrac{\partial\underline{\phi}}{\partial\underline{X}}\left(\underline{X}_1,t+\delta t\right)\cdot\underline{dX}.
			\end{aligned}
		\]
	Như vậy hai vecteur này có liên hệ với nhau bởi:
		\[
			\begin{aligned}			
				\underline{dx'}&=\dfrac{\partial\underline{\phi}}{\partial\underline{X}}\left(\underline{X}_1,t+\delta t\right)\cdot\underline{dX}\\
				&=\dfrac{\partial^2\underline{\phi}}{\partial\underline{X}\partial t}\left(\underline{X}_1,t\right)\cdot\underline{dX}\delta t\\
				&=\underline{\nabla_X}\left(\dfrac{\partial\underline{\phi}}{\partial t}\right)\left(\underline{X}_1,t\right)\cdot\underline{dX}\delta t
				% &=\underline{\underline{\nabla_X U}}\left(\underline{X}_1,t\right)\cdot\underline{dX}\delta t
			\end{aligned}
		\]
	Khi đưa vào định nghĩa gradient của trường vận tốc:
			\begin{align}
				\boxed{\underline{\underline{\nabla_X U}}\left(\underline{X},t\right):=\underline{\nabla_X}\left(\dfrac{\partial\underline{\phi}}{\partial t}\right)\left(\underline{X},t\right)}\ \text{khi}\ \underline{x}=\underline{\phi}\left(\underline{X},t\right).
			\end{align}
	Ta có thể biểu diển sự liên hệ giữa hai vecteur vật chất vô cùng bé bên trên:
		\[
			\begin{aligned}
				\underline{dx'}&=\underline{\underline{\nabla_X U}}\left(\underline{X},t\right)\cdot\underline{dX}\delta t\\
				&=\underline{\underline{\nabla_X U}}\left(\underline{X},t\right)\cdot\left(\underline{\underline{F}}^{-1}\left(\underline{X},t\right)\cdot\underline{dx}\right)\delta t.
			\end{aligned}
		\]
	Ta gọi đạo hàm vật chất của một vecteur vô cùng nhỏ đặt tại điểm $\underline{M}_1$ ở thời điểm $t$ bởi:
		\[
			\begin{aligned}
				\dot{\arc{\underline{dx}}}:=\lim\limits_{\delta t\rightarrow0}\dfrac{\underline{dx'}-\underline{dx}}{\delta t}=\boxed{\left(\underline{\underline{\nabla_X U}}\left(\underline{X},t\right)\cdot\underline{\underline{F}}^{-1}\left(\underline{X},t\right)\right)\cdot\underline{dx}}.
			\end{aligned}
		\]
	Khi đó, đạo hàm vật chất của một vecteur vật chất vô cùng bé được viết lại khi định nghĩa gradient của trường vận tốc trong hình thái $\kappa_t$:
		\begin{align}
			\boxed{\dot{\arc{\underline{dx}}}=\underline{\underline{\nabla u}}\left(\underline{x},t\right)\cdot\underline{dx}}.
		\end{align}
	trong đó:
		\begin{align}
			\boxed{\underline{\underline{\nabla u}}\left(\underline{x},t\right):=\underline{\underline{\nabla_X U}}\left(\underline{X},t\right)\cdot\underline{\underline{F}}^{-1}\left(\underline{X},t\right)}\ \text{khi}\ \underline{x}=\underline{\phi}\left(\underline{X},t\right).
		\end{align}
\subsection{Tốc độ giản nở của thể tích}
	Chọn một hệ tọa độ Descartes trực chuẩn $(O, \underline{e}_1,\underline{e}_2,\underline{e}_3)$ cố định trong không gian. Xét ba vecteur vật chất $\underline{dx}_1$, $\underline{dx}_2$ và $\underline{dx}_3$ không đồng phẳng đặt tại điểm $\underline{x}$ trong hình thái $\kappa_t$. Thể tích của hình lăng trụ được tạo bởi ba vecteur này được tính thông qua sự hỗ trợ của hệ tọa độ vừa chọn ở bên trên. Đầu tiên, ta định nghĩa tenseur thể tích\footnote{Đơn giản là việc sắp xếp các vecteur thành một ma trận vuông mà lần lượt mỗi vecteur vật chất tạo thành một cột của ma trận này. Chúng tôi không viết đơn giản, vì ở đây, chúng tôi muốn nhấn mạnh đếp quan điểm "thao tác".}:
		\[
			\underline{\underline{\mathscr{V}}}=\underline{dx}_1\otimes\underline{e}_1+\underline{dx}_2\otimes\underline{e}_2+\underline{dx}_3\otimes\underline{e}_3
		\]
	Nhờ vào nó, ta tính được thể tích của hình lăng trụ bên trên:
		\begin{align}
			d\Omega_t=\det\underline{\underline{\mathscr{V}}}.
		\end{align}

	Sau đó, ở thời điểm $t+\delta t$, các vecteur này lần lượt bị dịch chuyển thành ba vecteur vật chất mới $\underline{dx}'_1$, $\underline{dx}'_2$ và $\underline{dx}'_3$ đặt tại vị trí $\underline{x}'$ và ta tính tenseur thể tích tương tự như bên trên:
		\[
			\underline{\underline{\mathscr{v}}}=\underline{dx}'_1\otimes\underline{e}_1+\underline{dx}'_2\otimes\underline{e}_2+\underline{dx}'_3\otimes\underline{e}_3
		\]
	Nhờ vào nó, ta tính được thể tích của hình lăng trụ mới:
		\begin{align}
			d\Omega_{t+\delta t}=\det\underline{\underline{\mathscr{v}}}.
		\end{align}
	
	Dựa vào công thức đạo hàm hạt của vecteur vật chất đã tìm ra ở bên trên, ta có thể biến đổi:
		\[
			\begin{aligned}
				d\Omega_{t+\delta t}-d\Omega_{t}&=|\begin{array}{ccc}
					\underline{dx}'_{1} & \underline{dx}'_{2} & \underline{dx}'_{3}\end{array}|-|\begin{array}{ccc}\underline{dx}_{1} & \underline{dx}_{2} & \underline{dx}_{3}\end{array}|\\
					&=|\begin{array}{ccc}\underline{dx}'_{1}-\underline{dx}_{1} & \underline{dx}'_{2} & \underline{dx}'_{3}\end{array}|-|\begin{array}{ccc}\underline{dx}_{1} & \underline{dx}'_{2}-\underline{dx}_{2} & \underline{dx}'_{3}\end{array}|\\
					&\qquad\qquad\qquad\qquad\qquad+|\begin{array}{ccc}
					\underline{dx}_{1} & \underline{dx}_{2} & \underline{dx}'_{3}-\underline{dx}_{3}\end{array}|\\
					&=\delta t\left(|\begin{array}{ccc}\underline{\underline{\nabla u}}\cdot\underline{dx}_{1} & \underline{dx}'_{2} & \underline{dx}'_{3}\end{array}|+|\begin{array}{ccc}\underline{dx}_{1} & \underline{\underline{\nabla u}}\cdot\underline{dx}_{2} & \underline{dx}'_{3}\end{array}|\right.\\
					&\left.\qquad\qquad\qquad\qquad\qquad+|\begin{array}{ccc}\underline{dx}_{1} & \underline{dx}_{2} & \underline{\underline{\nabla u}}\cdot\underline{dx}_{3}\end{array}|\right)\\
			\end{aligned}
		\]
	Nếu như chỉ dừng lại ở các khai triển bậc một, ta có:
	\[
		\begin{aligned}
			d\Omega_{t+\delta t}-d\Omega_{t}&=\delta t\left(|\begin{array}{ccc}\underline{\underline{\nabla u}}\cdot\underline{dx}_{1} & \underline{dx}_{2} & \underline{dx}_{3}\end{array}|+|\begin{array}{ccc}\underline{dx}_{1} & \underline{\underline{\nabla u}}\cdot\underline{dx}_{2} & \underline{dx}_{3}\end{array}|\right.\\
			&\left.\qquad\qquad\qquad\qquad\qquad+|\begin{array}{ccc}\underline{dx}_{1} & \underline{dx}_{2} & \underline{\underline{\nabla u}}\cdot\underline{dx}_{3}\end{array}|\right)
		\end{aligned}
	\]
	Do đó tốc độ biến đổi thể tích 
		\[
			\begin{aligned}
				\lim\limits_{\delta t\rightarrow0}\dfrac{d\Omega_{t+\delta t}-d\Omega_{t}}{\delta t}&=|\begin{array}{ccc}\underline{\underline{\nabla u}}\cdot\underline{dx}_{1} & \underline{dx}_{2} & \underline{dx}_{3}\end{array}|+|\begin{array}{ccc}\underline{dx}_{1} & \underline{\underline{\nabla u}}\cdot\underline{dx}_{2} & \underline{dx}_{3}\end{array}|\\
				&\qquad\qquad\qquad\qquad\qquad+|\begin{array}{ccc}\underline{dx}_{1} & \underline{dx}_{2} & \underline{\underline{\nabla u}}\cdot\underline{dx}_{3}\end{array}|\\
				&=\underline{\underline{\nabla u}}\colon\underline{\underline{\mathbbm{1}}}d\Omega_t
			\end{aligned}
		\]
	Mà ta có thể viết gọn lại thành:
		\begin{align}
			\dot{\arc{d\Omega_t}}=\underline{\nabla}\cdot\underline{u}\left(\underline{x},t\right)d\Omega_t
		\end{align}

	Công thức này cho ta thấy vai trò của div của trường vận tốc, nó đại diện cho mức độ giản nở của lưu chất. Ta sẽ biểu diển lại vai trò này thông qua tóc độ biến đổi thể tích tương đối:
	\begin{align}
		\dfrac{\dot{\arc{d\Omega_t}}}{d\Omega_t}\left(\underline{x},t\right)=\underline{\nabla}\cdot\underline{u}\left(\underline{x},t\right)
	\end{align}
\subsection{Tốc độ xoay của lưu chất}

\end{document}