\documentclass[../../../main.tex]{subfiles}

\begin{document}
Hành vi của lưu chất là các ứng xử của lưu chất khi gặp phải các điều kiện ngoài, tức là các phản ứng nội tại của lưu chất khi gặp các tác động vật lý lên nó, chẳng hạn như sự thay đổi nhiệt độ, các lực được áp đặt lên lưu chất hoặc hình dạng của cố thể mà lưu chất tương tác khi nó chuyển động qua cố thể này.

Một cách tổng quát, trạng thái của lưu chất được định nghĩa thông qua các biến số vĩ mô nội tại của lưu chất; và thường là một hệ thức liên hệ các tham số động lực học và tham số động học của lưu chất, kết hợp với các thông số nhiệt động lực học để mô tả bản chất trao đổi năng lượng của lưu chất. Ở phần cân bằng các đại lượng cơ học, ta chỉ mới sử dụng dạng tổng quát của trạng thái ứng suất của lưu chất, còn dạng tường minh của nó là gì thì ta vẫn chưa đề cập. Như vậy, việc xác định được dạng tường minh trạng thái ứng suất của lưu chất sẽ giúp ta hoàn thành việc mô tả\footnote{Ít nhất là về mặt hình thức các ứng xử của lưu chất}. Do đó, bây giờ ta sẽ nghiên cứu một số dạng hành vi ứng xử điển hình của lưu chất.
\subsection{Lưu chất newton}

\end{document}