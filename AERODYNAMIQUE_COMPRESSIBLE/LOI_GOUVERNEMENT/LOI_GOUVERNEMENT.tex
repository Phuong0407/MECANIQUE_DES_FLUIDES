\documentclass[../DONG_CHAY_NEN_DUOC.tex]{subfiles}

\begin{document}
\chapter{CÁC PHƯƠNG TRÌNH CHI PHỐI}
	\section{Mở đầu}
		\documentclass[../../../main.tex]{subfiles}

\begin{document}
    Tất cả các dòng chuyển động của không khí đều là dòng chảy nén được, bởi vì không khí là nén được. Tuy nhiên, đối với các dòng chảy có tốc độ không quá lớn, theo \bi{thực nghiệm}, số Mach của nó phải thỏa $M <0.3$, ta có thể đơn giản xấp xỉ nó như các dòng lưu chất không nén được và áp dụng các kết quả đã biết đối với dòng chảy không nén được của lưu chất. Điều này có thể thực hiện được bởi vì ở các vận tốc chuyển động không quá lớn, tốc độ của các hiện tượng lan truyền là không quá lớn và do đó sự trao đổi năng lượng là không quá đáng kể. Điều này làm cho các xử lý liên quan đến năng lượng (có bản chất nhiệt động lực học) là không cần thiết.

    Tuy nhiên, đối với các dòng chuyển động có tốc độ lớn hơn, số Mach $M\ge 0.3$, những xử lý nhiệt động lực học là không thể tránh khỏi. Bên cạnh đó, các hiện tượng lan truyền sẽ xảy ra và một sự xử lý nó bằng toán học là cần thiết, điều này sẽ được thực hiện thông qua bài toán Riemann. Do đó, đầu tiên, chúng ta hãy nhắc lại về các khái niệm đã biết trong cơ lưu chất và nhiệt động lực học.
\end{document}
	\section{Nhắc lại về cơ lưu chất}
		Ở đây xin không chứng minh lại các biểu thức mà chỉ đơn giản liệt kê chúng với mục đích làm tham chiếu trực tiếp cho các phát triển sau này. Đầu tiên là phương trình bảo toàn lưu lượng:
	\begin{align}\label{eq:conserve_mass}
		\dfrac{\partial\rho}{\partial t}+\underline{\nabla}\cdot\left(\rho\underline{u}\right)=0.
	\end{align}
Tiếp theo là phương trình Navier-Stokes (phương trình cân bằng động lượng) : mặc dù có phương trình dưới dạng tổng quát, phương trình động lượng Cauchy, mà nó được viết
	\begin{align}
		\rho\dfrac{D\underline{u}}{Dt}=-\underline\nabla\cdot\underline{\underline{\sigma}}+\underline{f}.
	\end{align}
trong đó $\underline{\underline{\sigma}}$ là tenseur ứng suất, và $\underline{f}$ là lực thể tích tác dụng lên lưu chất. Như đã thảo luận, trong dòng chảy nén được, không thể bỏ qua hiện tượng lan truyền sóng, do dó, chúng tôi đưa ra phương trình tổng quát sau đây
	\begin{align}\label{eq:nse}
		\rho\dfrac{D\underline{u}}{Dt}=-\underline{\nabla}p+\vnabla\cdot\left[\mu\left(\vnabla\underline{u}+{}^t\vnabla\underline{u}-\dfrac{2}{3}\left(\vnabla\cdot\underline{u}\right)\underline{\underline{\mathbbm{1}}}\right)+\zeta\left(\vnabla\cdot\underline{u}\right)\underline{\underline{\mathbbm{1}}}\right]+\rho\underline{g}.
	\end{align}
trong đó $\zeta=\lambda+2\mu/3$ là hệ số nhớt khối, là một hệ số phụ thuộc không những vào đặc tính lưu chất mà còn vào đặc tính dòng chuyển động. Điều này ứng với phương trình ứng suất-biến dạng có dạng:
	\begin{align}
		\underline{\underline{\tau}}=\mu\left(\vnabla\underline{u}+{}^t\vnabla\underline{u}-\dfrac{2}{3}\left(\vnabla\cdot\underline{u}\right)\underline{\underline{\mathbbm{1}}}\right)+\zeta\left(\vnabla\cdot\underline{u}\right)\underline{\underline{\mathbbm{1}}}.
	\end{align}
Các hệ số nhớt này biến đổi thế nào theo nhiệt độ? Tuy nhiên trong các nghiên cứu mà không có sự hấp thụ âm thanh hay sự suy giảm sóng xung kích, ta sẽ sử dụng dạng phương trình đơn giản hơn của phương trình Navier-Stokes:
	\begin{align}
		\boxed{
			\rho\dfrac{D\underline{u}}{Dt}=-\underline{\nabla}p+\mu\Delta\underline{u}+\dfrac{1}{3}\mu\underline{\nabla}\left(\underline{\nabla}\cdot\underline{u}\right)+\rho\underline{g}.
		}
	\end{align}
	\section{Nhắc lại về nhiệt động lực học}
		\documentclass[../../../main.tex]{subfiles}

\begin{document}
Xét miền $\Omega$ có biên $\partial\Omega$ là đủ chính quy được lấp đầy bởi một lưu chất nén được.
\subsection{Cân bằng năng lượng tổng quát}
    Nguyên lý thứ nhất của nhiệt động lực học phát biểu rằng, năng lượng của lưu chất được bảo toàn. Năng lượng của lưu chất bao gồm nội năng và động năng, mà khi xem xét sự biến đổi, phải bằng tổng lượng nhiệt và lượng công mà lưu chất trao đổi (cho và nhận), như vậy nếu kí hiệu $e$ là nội năng riêng của lưu chất, phương trình cân bằng năng lượng được viết:
        \[
            \dfrac{D}{Dt}\iiint_\mathscr{V}\rho\left(e+\dfrac{\underline{u}^2}{2}\right)d\tau=\dot{W}+\dot{Q}.
            \]
    Nhiệt mà lưu chất trao đổi bao gồm lượng nhiệt mà bản thân lưu chất sinh ra và không có nguồn gốc cơ học vĩ mô, chẳng hạn khi có sự xuất hiện của một phản ứng hóa học và thông qua sự truyền nhiệt với môi trường bên ngoài. Công mà lưu chất trao đổi bao gồm công do các tác động cơ ngoại sinh ra và công do chính các tác động cơ nội bên trong lưu chất. Do đó, nếu gọi $q$ là tốc độ sinh nhiệt riêng của lưu chất và $\underline{j}_{th}$ là vecteur mật độ dòng nhiệt (theo quy ước, luôn luôn hướng ra khỏi $\Omega$), sự trao đổi nhiệt có thể được viết:
    \[
            \begin{aligned}
                \dot{Q}&=\iiint_\mathscr{V}\rho qd\tau+\oiint_\mathscr{S} -\underline{j}_{th}d\underline{S},\\
                \dot{W}&=\iiint_\mathscr{V}\underline{\underline{\sigma}}:\underline{\underline{D}}d\tau+\iiint_\mathscr{V}\rho \underline{g}\cdot\underline{u}d\tau
            \end{aligned}
        \]
    trong đó
        \[
            \underline{\underline{D}}=\dfrac{1}{2}\left(\underline{\nabla}\underline{u} +{}^t\underline{\nabla}\underline{u}\right)
        \]
    là tenseur tốc độ biến dạng. Kết hợp các phương trình này lại, sử dụng định lý Gauss-Odtrogradsky và công thức đạo hàm đối lưu đối với đại lượng thể tích, ta có phương trình cân bằng năng lượng dưới dạng vi phân:
    \begin{align}
        \dfrac{\partial}{\partial t}\left[\rho\left(e+\dfrac{\underline{u}^2}{2}\right)\right]+\underline{\nabla}\cdot\left[\rho\left(e+\dfrac{\underline{u}^2}{2}\right)\underline{u}\right]=\rho q-\underline{\nabla}\cdot\underline{j}_{th}+\underline{\underline{\sigma}}:\underline{\underline{D}}+\rho\underline{g}\cdot\underline{u}.
    \end{align}
    Sử dụng định nghĩa của tenseur tốc độ biến dạng, ta có:
        \begin{align}\label{eq:conserver_energie}
            \dfrac{\partial}{\partial t}\left[\rho\left(e+\dfrac{\underline{u}^2}{2}\right)\right]+\underline{\nabla}\cdot\left[\rho\left(e+\dfrac{\underline{u}^2}{2}\right)\underline{u}+\underline{j}_{th}-\underline{\underline{\sigma}}\cdot\underline{u}\right]=\rho q+\rho\underline{g}\cdot\underline{u}.
        \end{align}
\subsection{Cân bằng entropy}
    Bây giờ ta áp dụng các khái niệm đã biết của nguyên lý thứ hai nhiệt động lực học cho khối lưu chất. Nếu gọi $s$ là entropy riêng của lưu chất, thế thì entropy của toàn bộ khối lưu chất được viết 
        \[
            S=\iiint_\mathscr{V}\rho sd\tau.
            \]
            Entropy liên hệ trực tiếp đến thông tin của hệ thống, do đó nó không thể bị phá hủy, điều đó chứng tỏ phải có sự cân bằng entropy. Sự biến thiên entropy của lưu chất có thể do sự cung cấp của môi trường bên ngoài và sự biến đổi của tự bản thân lưu chất, nếu gọi $\underline{\Phi}_S$ là vecteur thông lượng entropy sinh ra do tương tác với môi trường bên ngoài (được quy ước hướng ra ngoài $\Omega$) và $q_s$ là tốc độ sinh ra entropy riêng bên trong bản thân lưu chất, ta có:
            \[
                \Delta S=\iiint_\mathscr{V}\rho q_sd\tau+\oiint_\mathscr{S}-\underline{\Phi}_Sd\underline{S}.
                \]
    Như vậy entropy nội sinh của lưu chất được tính :
    \begin{align}
        S_{\text{ns}}=\dot{S}-\Delta S=\dfrac{D}{Dt}\iiint_\mathscr{V} \rho sd\tau-\iiint_\mathscr{V}\rho q_sd\tau+\oiint_\mathscr{S}\underline{\Phi}_Sd\underline{S}.
        \end{align}
        Theo nguyên lý thứ hai nhiệt động lực học, $S_{\text{ns}}\ge0$, do đó khi sử dụng định lý Gauss-Odtrogradsky, ta có bất đẳng thức entropy cục bộ:
        \begin{align}
            \dfrac{D}{Dt}(\rho s)-\rho q_s+\underline{\nabla}\cdot\underline{\Phi}_S\ge 0.
        \end{align}
    Sử dụng các mật độ trao đổi nhiệt trong phần trên, ta có:
    \begin{align}
        \boxed{
            \dfrac{D(\rho s)}{Dt} - \rho\dfrac{q}{T} + \underline{\nabla}\cdot\left(\dfrac{\underline{j}_{th}}{T}\right)\ge 0
            }.
        \end{align}
    Đây là một bất đẳng thức quan trọng và được gọi là \bi{bất đẳng thức Claussius-Duhem}, bởi vì, mọi hành vi của lưu chất mà không thỏa mãn bất đẳng thức này đều không được phép xảy ra.
\end{document}
	\section{Hệ phương trình chi phối hành vi của lưu chất}
		\documentclass[../../../main.tex]{subfiles}

\begin{document}
    Như đã đề cập ở bên trên, một lưu chất nén được được đặc trưng thông qua khối lượng riêng, trường vận tốc, trường nhiệt độ và trường áp suất. Tức là 6 thông số cần phải được mô tả để mô tả đặc tính của một lưu chất nén được. Các phương trình (\ref{eq:conserve_mass}), (\ref{eq:nse}) và (\ref{eq:conserver_energie}) chỉ cung cấp 5 phương trình, như vậy là còn thiếu một phương trình.

    Phương trình còn thiếu này chắc chắn phải liên quan đến hành vi của lưu chất, mà ta sẽ gọi là phương trình trạng thái của lưu chất. Có nhiều phương trình trạng thái, nhưng có hai phương trình quan trọng và sẽ được sử dụng trong toàn bộ phần nghiên cứu này:
        \begin{itemize}
            \item Phương trình trạng thái của khí lý tưởng:
                \begin{align}
                    p = \rho RT.
                \end{align}
            Trong đó $R=\SI{287.07}{\joule\per\kilogram\per\kelvin}$ là hằng số khí cho không khí. Đối với các khí khác, hằng số được cho trong bảng (???).
            \item Phương trình trạng thái polytrophic: một khí polytrophic là một khí lý tưởng mà nhiệt dung riêng đẳng tích là hằng số, tức là
                \begin{align}
                    e = c_vT
                \end{align}
            trong đó $c_v$ là nhiệt dung riêng đẳng tích và là hằng số. Tỉ số giữa nhiệt dung riêng đẳng áp và nhiệt dung riêng đẳng tích hiển nhiên là hằng số, đối với không khí
                \begin{align}
                    \gamma = \dfrac{c_p}{c_v}=1.4.
                    % \begin{cases}
                    %     1.4 & \text{đối với các điều kiện nhiệt độ thông thường}\\
                    %     1.3 & \text{đối với các buồng đốt có nhiệt độ cao}
                    % \end{cases}.
                \end{align}
        \end{itemize}

    Như vậy, trong khuôn khổ của phần lớn nghiên cứu của chúng ta, ta chỉ làm việc với khí lý tưởng polytropic. Hơn nữa, quá trình trao đổi nhiệt được giới hạn trong định luật truyền nhiệt \textsc{Fourier}:
    	\begin{align}
    		\underline{j}_{th}=-K\underline{\nabla T}.
    	\end{align}
    Trong đó $K$ được gọi là hệ số dẫn nhiệt. Khi đó, hệ phương trình chi phối hành vi của lưu chất được viết:
        \begin{equation}\label{eq:reel}
            \boxed{
                \begin{aligned}
                    &\dfrac{\partial\rho}{\partial t}+\underline{\nabla}\cdot\left(\rho\underline{u}\right)=0\\[10pt]
                    &\rho\frac{D\underline{u}}{Dt}=-\underline{\nabla}p+\mu\Delta\underline{u}-\dfrac{1}{3}\mu\underline{\nabla}\cdot\left(\underline{\nabla}\cdot\underline{u}\right)+\rho\underline{g}\\[10pt]
                    &\dfrac{\partial}{\partial t}\left[\rho\left(c_vT+\dfrac{\underline{u}^2}{2}\right)\right]+\underline{\nabla}\cdot\left[\rho\left(c_vT+\dfrac{\underline{u}^2}{2}\right)\underline{u}+\underline{j}_{th}-\underline{\underline{\sigma}}\cdot\underline{u}\right]=\rho q+\rho\underline{g}\cdot\underline{u}\\[10pt]
                    &p=\rho RT
                \end{aligned}
            },
        \end{equation}
    mà ta gọi là \bi{hệ phương trình lưu chất thực}.

    Trong chừng mực mà lưu chất được xem là không nhớt, không có sự trao đổi nhiệt, các lực thể tích có thể bỏ qua được thì hệ phương trình trên được đơn giản thành hệ phương trình có phương trình động lượng theo Euler:
        \begin{equation}\label{eq:parfait}
            \boxed{
                \begin{aligned}
                    &\dfrac{\partial\rho}{\partial t}+\underline{\nabla}\cdot\left(\rho \underline{u}\right)=0\\[10pt]
                    &\rho\dfrac{D\underline{u}}{Dt}=-\underline{\nabla}p\\[10pt]
                    &\dfrac{\partial}{\partial t}\left[\rho\left(c_vT+\dfrac{ \underline{u}^2}{2}\right)\right]+\underline{\nabla}\cdot\left[\rho\left(c_vT+\dfrac{\underline{u}^2}{2}\right)\underline{u} \right]=0\\[10pt]
                    &p=\rho RT
                \end{aligned}
	        },
        \end{equation}
    mà ta gọi là \bi{hệ phương trình lưu chất lý tưởng}.

    \begin{description}
        \item[Chú ý:] Tồn tại nhiền hơn phương trình trạng thái của lưu chất bên cạnh phương trình khí lý tưởng. Chẳng hạn như phương trình Wan der Waals dành cho khí thực:
            \[
                \left(P+\dfrac{n^2a}{V}\right)\left(V-nb\right)=nRT.
            \]
        % \item[Chú ý 2:] Ta hãy khử thứ nguyên của phương trình (\ref{eq:reel}) tùy theo điều kiện của miền mà lưu chất chiếm mà miền này có:
        % \begin{center}
        %     \begin{tabular}{||c|c|c||}
        %         \hline
        %             Đại lượng & Kí hiệu (đặc trưng) & Kí hiệu (vô thứ nguyên)\tabularnewline
        %         \hline
        %         \hline
        %             Chiều dài & $L_{\infty}$ & $x^*=x/L$ \tabularnewline
        %         \hline
        %             Vận tốc & $U_{\infty}$ & $t^*=t/t_0$ \tabularnewline
        %         \hline
        %             Khối lượng riêng & $\rho_{\infty}$ & $p^*=p/p_\infty$ \tabularnewline
        %         \hline
        %             Áp suất & $p_{\infty}$ & $\underline{u}^*=\underline{u}/U_\infty$ \tabularnewline
        %         \hline
        %             Độ nhớt & $\mu_{\infty}$ & $\mu^*=\mu/\mu_\infty$ \tabularnewline
        %         \hline
        %             Hệ số dẫn nhiệt & $k_{\infty}$ & $k^*=k/k_\infty $\tabularnewline
        %         \hline
        %     \end{tabular}
        % \end{center}
        % Trong đó ta giả sử rằng sự truyền nhiệt thỏa mãn định luật Fourier:
        %     \[
        %         \underline{j}_{th}=-k\underline{\nabla}T.
        %     \]
        % Như vậy, ta có thể định nghĩa thời gian tham chiếu như là
        %     \[
        %         t_0 =\dfrac{L}{U_\infty}.
        %     \]
        % % và các biến số vô thứ nguyên mới:
        % %     \[
        % %         x^*=\dfrac{x}{L},\quad t^*=\dfrac{t}{t_0},\quad\rho^*=\dfrac{\rho}{\rho_\infty},\quad p^*=\dfrac{p}{p_\infty},\quad\underline{u}^*=\dfrac{\underline{u}}{U_\infty},\quad\mu^*=\dfrac{\mu}{\mu_0},\quad k^*=\dfrac{k}{k_0}.
        % %     \]
        % Hệ phương trình (\ref{eq:reel}) được biến đổi thành:
        %     \[
        %         \begin{aligned}
        %             \begin{cases}
        %                 \dfrac{\partial\rho^*}{\partial t^*}+\underline{\nabla}^*\cdot\left(\rho^*\underline{u}^*\right)=0\\[10pt]
        %                 \rho^*\dfrac{D^*\underline{u}^*}{Dt^*}=-\dfrac{\gamma}{M^2_\infty}\underline{\nabla}^* p^*+\dfrac{1}{Re} \underline{\nabla}^*\cdot\left[\mu^*\left(\underline{\nabla}^* \underline{u}^*+{}^t\underline{\nabla}^* \underline{u}^*-\dfrac{2}{3}\left(\underline{\nabla}^*\cdot\underline{u}^*\right) \underline{\underline{\mathbbm{1}}}\right)\right]&\\
        %                 \qquad\qquad\qquad+\dfrac{1}{\rho_\infty U_\infty^2L} \underline{\nabla}\cdot\left(\zeta\left(\underline{\nabla}\cdot\underline{u}\right)\underline{\underline{\mathbbm{1}}}\right)+\dfrac{1}{Fr^2}\underline{e}_g&\\[10pt]
        %                 \dfrac{\partial^*}{\partial t^*}\left[\rho^*\left(c_vT+\dfrac{(\underline{u}^*)^2}{2}\right)\right]+\underline{\nabla}^*\cdot\left[\rho^*\left(c_vT+\dfrac{\underline{u}^2}{2}\right)\underline{u}^*+\underline{j}_{th}-\underline{\underline{\sigma}}\cdot\underline{u}\right]=\rho q + \rho\underline{g}\cdot\underline{u}\\[10pt]
        %                 p = \rho RT
        %             \end{cases}
        %         \end{aligned}
        %     \]
        % trong đó $Re$ là số Reynolds và $Fr$ là số Froude.
        
        % Điều này dẫn chúng ta đến một sự đơn giản chớp nhoáng, trong đó khi số Reynolds là rất lớn, số hạng lực nhớt có thể bị bỏ qua và ta thu được phương trình động lượng Euler cho lưu chất lý tưởng. Phải chăng sự gần đúng này là hợp lý ? Một cách ngây thơ, chỉ cần bỏ đi số hạng nhớt, tuy nhiên, hai phương trình động lượng Euler và Navier-Stokes biểu hiện hoàn toàn khác nhau. Thực vậy, nghiệm của phương trình Euler là hoàn toàn trơn. Còn đối với PNS ở số Reynolds rất lớn, nghiệm luôn luôn thể hiện sự mất ổn định.
    \end{description}
\end{document}
	\section{Tính đến sự bất liên tục}
Giả sử tồn tại một mặt không liên tục, $\Sigma_t$ trong miền $\Omega_t$. Như những điều đã biết thì đối với một đại lượng thể tích $g$ ($g$ có thể là một đại lượng vô hướng, vecteur hoặc tenseur), đạo hàm đối lưu đối với đại lượng này như sau:
\begin{align}
	\dfrac{D}{Dt}\iiint_{\Omega_t}gd\tau=\iiint_{\Omega_t}\left(\dfrac{\partial g}{\partial t}+\underline{\nabla}\cdot\left(g\otimes\underline{U}\right)\right)d\tau+\oiint_{\Sigma_t}\llbracket g\left(\underline{U}-\underline{W}\right)\rrbracket dS
\end{align}

Áp dụng biểu thức này đối với khối lượng riêng, $\rho$, ta thu được phương trình liên hệ cho sự không liên tục của khối lượng riêng:
\begin{align}
	\llbracket\rho\left(\underline U-\underline W\right)\rrbracket=0.
\end{align}











\end{document}