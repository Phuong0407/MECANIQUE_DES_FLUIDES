\documentclass[../../../main.tex]{subfiles}

\begin{document}
    Tất cả các dòng chuyển động của không khí đều là dòng chảy nén được, bởi vì không khí là nén được. Tuy nhiên, đối với các dòng chảy có tốc độ không quá lớn, theo \bi{thực nghiệm}, số Mach của nó phải thỏa $M <0.3$, ta có thể đơn giản xấp xỉ nó như các dòng lưu chất không nén được và áp dụng các kết quả đã biết đối với dòng chảy không nén được của lưu chất. Điều này có thể thực hiện được bởi vì ở các vận tốc chuyển động không quá lớn, tốc độ của các hiện tượng lan truyền là không quá lớn và do đó sự trao đổi năng lượng là không quá đáng kể. Điều này làm cho các xử lý liên quan đến năng lượng (có bản chất nhiệt động lực học) là không cần thiết.

    Tuy nhiên, đối với các dòng chuyển động có tốc độ lớn hơn, số Mach $M\ge 0.3$, những xử lý nhiệt động lực học là không thể tránh khỏi. Bên cạnh đó, các hiện tượng lan truyền sẽ xảy ra và một sự xử lý nó bằng toán học là cần thiết, điều này sẽ được thực hiện thông qua bài toán Riemanne. Do đó, đầu tiên, chúng ta hãy nhắc lại về các khái niệm đã biết trong cơ lưu chất và nhiệt động lực học.
\end{document}