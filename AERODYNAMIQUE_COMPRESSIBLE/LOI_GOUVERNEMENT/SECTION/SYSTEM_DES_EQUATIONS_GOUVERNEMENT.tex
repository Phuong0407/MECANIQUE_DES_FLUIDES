\documentclass[../../../main.tex]{subfiles}

\begin{document}
    Như đã đề cập ở bên trên, một lưu chất nén được được đặc trưng thông qua khối lượng riêng, trường vận tốc, trường nhiệt độ và trường áp suất. Tức là 6 thông số cần phải được mô tả để mô tả đặc tính của một lưu chất nén được. Các phương trình (\ref{eq:conserve_mass}), (\ref{eq:nse}) và (\ref{eq:conserver_energie}) chỉ cung cấp 5 phương trình, như vậy là còn thiếu một phương trình.

    Phương trình còn thiếu này chắc chắn phải liên quan đến hành vi của lưu chất, mà ta sẽ gọi là phương trình trạng thái của lưu chất. Có nhiều phương trình trạng thái, nhưng có hai phương trình quan trọng :
        \begin{itemize}
            \item Phương trình trạng thái của khí lý tưởng:
                \begin{align}
                    p = \rho RT.
                \end{align}
            Trong đó $R=\SI{287.07}{\joule\per\kilogram\per\kelvin}$ là hằng số khí cho không khí. Đối với các khí khác, hằng số được cho trong bảng (???).
            \item Phương trình trạng thái polytrophic: một khí polytrophic là một khí lý tưởng mà nhiệt dung riêng là hằng số, tức là
                \begin{align}
                    e = c_vT
                \end{align}
            trong đó $c_v$ là nhiệt dung riêng đẳng tích và là hằng số. Tỉ số giữa nhiệt dung riêng đẳng áp và nhiệt dung riêng đẳng tích là hằng số, đối với không khí
                \begin{align}
                    \gamma = \dfrac{c_p}{c_v} = \begin{cases}
                        1.4 & \text{đối với các điều kiện nhiệt độ thông thường}\\
                        1.3 & \text{đối với các buồng đốt có nhiệt độ cao}
                    \end{cases}.
                \end{align}
        \end{itemize}

    Như vậy, trong khuôn khổ của phần lớn nghiên cứu của chúng ta, ta chỉ làm việc với khí lý tưởng polytropic. Do đó, hệ phương trình chi phối hành vi của lưu chất được viết:
        \begin{equation}\label{eq:reel}
            \boxed{
                \begin{aligned}
                    &\dfrac{\partial\rho}{\partial t}+\underline{\nabla}\cdot\left(\rho\underline{u}\right)=0\\[10pt]
                    &\rho\frac{D\underline{u}}{Dt}=-\underline{\nabla}p+\mu\Delta\underline{u}-\dfrac{1}{3}\mu\underline{\nabla}\cdot\left(\underline{\nabla}\cdot\underline{u}\right)+\rho\underline{g}\\[10pt]
                    &\dfrac{\partial}{\partial t}\left[\rho\left(c_vT+\dfrac{\underline{u}^2}{2}\right)\right]+\underline{\nabla}\cdot\left[\rho\left(c_vT+\dfrac{\underline{u}^2}{2}\right)\underline{u}+\underline{j}_{th}-\underline{\underline{\sigma}}\cdot\underline{u}\right]=\rho q+\rho\underline{g}\cdot\underline{u}\\[10pt]
                    &p=\rho RT
                \end{aligned}
            },
        \end{equation}
    mà ta gọi là \bi{hệ phương trình lưu chất thực}.

    Trong chừng mực mà lưu chất được xem là không nhớt, không có sự trao đổi nhiệt, cũng như các lực thể tích hệ phương trình trên được đơn giản thành hệ phương trình có phương trình động lượng theo Euler:
        \begin{equation}\label{eq:parfait}
            \boxed{
                \begin{aligned}
                    &\dfrac{\partial\rho}{\partial t}+\underline{\nabla}\cdot\left(\rho \underline{u}\right)=0\\[10pt]
                    &\rho\dfrac{D\underline{u}}{Dt}=-\underline{\nabla}p\\[10pt]
                    &\dfrac{\partial}{\partial t}\left[\rho\left(c_vT+\dfrac{ \underline{u}^2}{2}\right)\right]+\underline{\nabla}\cdot\left[\rho\left(c_vT+\dfrac{\underline{u}^2}{2}\right)\underline{u} \right]=0\\[10pt]
                    &p=\rho RT
                \end{aligned}
	        },
        \end{equation}
    mà ta gọi là \bi{hệ phương trình lưu chất lý tưởng}.

    \begin{description}
        \item[Chú ý 1:] Tồn tại nhiền hơn phương trình trạng thái của lưu chất bên cạnh phương trình khí lý tưởng. Chẳng hạn như phương trình Wan der Waals dành cho khí thực:
            \[
                \left(P+\dfrac{n^2a}{V}\right)\left(V-nb\right)=nRT.
            \]
        \item[Chú ý 2:] Ta hãy khử thứ nguyên của phương trình (\ref{eq:reel}) tùy theo điều kiện của miền mà lưu chất chiếm mà miền này có:
        \begin{center}
            \begin{tabular}{||c|c|c||}
                \hline
                    Đại lượng & Kí hiệu (đặc trưng) & Kí hiệu (vô thứ nguyên)\tabularnewline
                \hline
                \hline
                    Chiều dài & $L_{\infty}$ & $x^*=x/L$ \tabularnewline
                \hline
                    Vận tốc & $U_{\infty}$ & $t^*=t/t_0$ \tabularnewline
                \hline
                    Khối lượng riêng & $\rho_{\infty}$ & $p^*=p/p_\infty$ \tabularnewline
                \hline
                    Áp suất & $p_{\infty}$ & $\underline{u}^*=\underline{u}/U_\infty$ \tabularnewline
                \hline
                    Độ nhớt & $\mu_{\infty}$ & $\mu^*=\mu/\mu_\infty$ \tabularnewline
                \hline
                    Hệ số dẫn nhiệt & $k_{\infty}$ & $k^*=k/k_\infty $\tabularnewline
                \hline
            \end{tabular}
        \end{center}
        Trong đó ta giả sử rằng sự truyền nhiệt thỏa mãn định luật Fourier:
            \[
                \underline{j}_{th}=-k\underline{\nabla}T.
            \]
        Như vậy, ta có thể định nghĩa thời gian tham chiếu như là
            \[
                t_0 =\dfrac{L}{U_\infty}.
            \]
        % và các biến số vô thứ nguyên mới:
        %     \[
        %         x^*=\dfrac{x}{L},\quad t^*=\dfrac{t}{t_0},\quad\rho^*=\dfrac{\rho}{\rho_\infty},\quad p^*=\dfrac{p}{p_\infty},\quad\underline{u}^*=\dfrac{\underline{u}}{U_\infty},\quad\mu^*=\dfrac{\mu}{\mu_0},\quad k^*=\dfrac{k}{k_0}.
        %     \]
        Hệ phương trình (\ref{eq:reel}) được biến đổi thành:
            \[
                \begin{aligned}
                    \begin{cases}
                        \dfrac{\partial\rho^*}{\partial t^*}+\underline{\nabla}^*\cdot\left(\rho^*\underline{u}^*\right)=0\\[10pt]
                        \rho^*\dfrac{D^*\underline{u}^*}{Dt^*}=-\dfrac{\gamma}{M^2_\infty}\underline{\nabla}^* p^*+\dfrac{1}{Re} \underline{\nabla}^*\cdot\left[\mu^*\left(\underline{\nabla}^* \underline{u}^*+{}^t\underline{\nabla}^* \underline{u}^*-\dfrac{2}{3}\left(\underline{\nabla}^*\cdot\underline{u}^*\right) \underline{\underline{\mathbbm{1}}}\right)\right]&\\
                        \qquad\qquad\qquad+\dfrac{1}{\rho_\infty U_\infty^2L} \underline{\nabla}\cdot\left(\zeta\left(\underline{\nabla}\cdot\underline{u}\right)\underline{\underline{\mathbbm{1}}}\right)+\dfrac{1}{Fr^2}\underline{e}_g&\\[10pt]
                        \dfrac{\partial^*}{\partial t^*}\left[\rho^*\left(c_vT+\dfrac{(\underline{u}^*)^2}{2}\right)\right]+\underline{\nabla}^*\cdot\left[\rho^*\left(c_vT+\dfrac{\underline{u}^2}{2}\right)\underline{u}^*+\underline{j}_{th}-\underline{\underline{\sigma}}\cdot\underline{u}\right]=\rho q + \rho\underline{g}\cdot\underline{u}\\[10pt]
                        p = \rho RT
                    \end{cases}
                \end{aligned}
            \]
        trong đó $Re$ là số Reynolds và $Fr$ là số Froude.
        
        Điều này dẫn chúng ta đến một sự đơn giản chớp nhoáng, trong đó khi số Reynolds là rất lớn, số hạng lực nhớt có thể bị bỏ qua và ta thu được phương trình động lượng Euler cho lưu chất lý tưởng. Phải chăng sự gần đúng này là hợp lý ? Một cách ngây thơ, chỉ cần bỏ đi số hạng nhớt, tuy nhiên, hai phương trình động lượng Euler và Navier-Stokes biểu hiện hoàn toàn khác nhau. Thực vậy, nghiệm của phương trình Euler là hoàn toàn trơn. Còn đối với PNS ở số Reynolds rất lớn, nghiệm luôn luôn thể hiện sự mất ổn định.
    \end{description}
\end{document}