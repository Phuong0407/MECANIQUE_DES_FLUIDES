\documentclass[../DONG_CHAY_NEN_DUOC.tex]{subfiles}

\begin{document}
\chapter{DÒNG CHẢY NÉN ĐƯỢC DƯỚI ÂM QUA BIÊN DẠNG CÁNH}
    Các nghiên cứu về biên dạng cánh mà chúng ta đã thực hiện được thực hiện trong khuôn khổ dòng chảy không nén được. Tuy nhiên, một máy bay hiện đại không bao giờ bay ở số Mach thấp (dưới 0.3, mà ứng với nó vận tốc máy bay vào cở $\SI{400}{\km\per\hour}$), thực tế các máy bay dân sự thường bay ở vận tốc cận âm. Như vậy, các đặc tính của biên dạng cánh đã nghiên cứu có thể không còn phù hợp, do đó chương này sẽ thực hiện nghiên cứu lại các đặc tính khí động của biên dạng cánh trong trường hợp dòng chảy cận âm (đương nhiên là không thể bỏ qua đặc tính nén được).
    \section{Phương trình thế vận tốc}
        \subfile{./SECTION/EQUATION_DE_POTENTIEL_VELOCITE.tex}
    \section{Tuyến tính hóa phương trình thế vận tốc vô hướng}
        \subfile{./SECTION/LINEARISATION_DEQUATION_POTENTIEL.tex}
    \section{Sử dụng phương trình thế vận tốc đã tuyến tính hóa}
        
\end{document}