\documentclass[DONG_CHAY_NEN_DUOC.tex]{subfiles}

\begin{document}
\chapter{CÁC ĐỊNH LUẬT BẢO TOÀN}

\section{Các định luật bảo toàn}

%Xét $X$ là một tập mở của $\mathbb R^n$. Một miền chính quy trong $X$ là một tập con mở bị chặn của $X$ với điều kiện biên Lipschitz\footnote{Hiểu đơn giản là một miền mà biên của nó đủ chính quy}.

%Các định luật bảo toàn mà chúng ta xem xét ở đây là các định luật cân bằng sự tạo thành một hàm số nào đó trong một miền với thông lượng của hàm này trên bề mặt của nó. Do đó, ta gọi một định luật bảo toàn được hình thức hóa như sau:

%sự tạo thành bên trong miền $\mathcal D$ được cho bởi giá trị của một độ đo Radon $\mathscr{P}(\dom D)$ trong $X$. Xét trên biên $\bdom D$, tồn tại một hàm tập cộng tính $\mathscr{Q}_\dom{D}$, định nghĩa trên các tập con Borel của tập 







\end{document}