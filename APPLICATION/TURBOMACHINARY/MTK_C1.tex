\documentclass[MAY_THUY_KHI.tex]{subfiles}

\begin{document}
\chapter{CÁC Ý TƯỞNG CƠ BẢN}
\newpage
\section{Cánh chong chóng và dòng chảy}
        Khi nghiên cứu về cánh chong chóng của một máy nén, chúng ta sẽ không thể áp dụng các khái niệm của khí động lực học cánh cố định một cách dể dàng. Điều này là bởi vì dòng chuyển động của lưu chất dọc theo sải cánh là lớn, cùng với điểm cuối của cánh chong chóng là thành máy nén.
        
        Số lượng cánh chong chóng là nhiều trong một tầng của máy nén và do đó mỗi cánh có một ảnh hưởng không quá lớn lên đặc tính của dòng chảy lưu chất.

        Đối với một máy lưu chất hướng trục, góc của dòng chảy đi ra $\alpha_2$ và góc hướng ra của cánh chong chóng $\beta_2$ là khác nhau và ta gọi hiệu số của cái trước trừ cái sau là góc lệch, $\delta$:
            \begin{align}
                \delta=\alpha_2-\beta_2.
            \end{align}

        Một thông số khác quan trọng đối với bánh công tác đó là \bi{hệ số trượt} $\sigma$, mà ta định nghĩa như là tỉ số giữa vận tốc tiếp tuyến tuyệt đối trung bình đi ra khỏi bánh công tác và vận tốc tiếp tuyến tuyệt đối nếu vận tốc dòng ra là đều và đi theo hướng ra của cánh chong chóng. Cả hai thông số góc lệch và hệ số trượt là sự ảnh hưởng một cách thống trị của tính không nhớt, và khi đó lớp biên là mỏng. Nếu tình huống có tính nhớt, ta sẽ nghiên cứu kỹ hơn ở các phần sau. Nếu lớp biên là mỏng, điều kiện Kutta có thể được áp dụng và ta có thể xác định được lực tác dụng lên cánh chong chóng.
\section{Công nhập lượng}
        Áp suất và ứng suất trượt trên bánh công tác tạo ra một moment của bánh công tác so với trục của nó. Tích phân sự phân bố của chúng cho phép ta xác định được công nhập lượng vào máy nén, tuy nhiên, điều này là không thể về mặt thực hành. Do đó ta sẽ tính nó thông qua công suất nhập lượng và công suất xuất lượng của máy nén. Mặc dù điều này giúp ta bỏ qua được bản chất dòng chuyển động, tuy nhiên, chúng chỉ có thể áp dụng cho các dòng chảy không nén.

        Xét một dòng lưu chất có lưu lượng khối $\dot{m}$ đi vào ở bán kính $r_1$ với vận tốc tiếp tuyến $V_{\theta_1}$ và rời đi ở bán kính $r_2$ với vận tốc tiếp tuyến $ V_{\theta_2}$, do đó moment tác dụng lên trục quay được tính:
            \begin{align}
                T=\dot{m}(r_2 V_{\theta_2}-r_1 V_{\theta_1}).
            \end{align}
        Và công nhập lượng trên một đơn vị lưu lượng khối:
            \begin{align}
                \mathscr{P}=\omega(r_2 V_{\theta_2}-r_1 V_{\theta_1}).
            \end{align}
        Công thức này được gọi là \bi{phương trình Euler} cho máy lưu chất. Ta có thể viết công suất này với sự tham gia của vận tốc cánh chong chóng:
            \begin{align}
                \mathscr{P}=(U_2V_{\theta_2}-U_1V_{\theta_1}),
            \end{align}
        nếu dòng chuyển động đi vào và đi ra với cùng một bán kính, ta viết lại công thức này một cách đơn giản như sau:
            \begin{align}
                \mathscr{P}=U(V_{\theta_2}-V_{\theta_1}),
            \end{align}
        Phương trình Euler đúng bất kể moment động lượng được tạo ra, thậm chí cả lực cản ma sát trên cánh chong chóng tạo ra một công suất dương. Điều này là đúng bởi vì ứng suất tiếp là nhỏ so với sự biến thiên động lượng thẳng. Chỉ có một trường hợp mà công thức này không áp dụng được, đó là trường hợp xử lý thành để làm chậm sự stall, mà ở đó ứng suất trên thành là lớn. Từ đó, lực tiếp tuyến trên đơn vị khối lượng lưu chất được tính theo công thức:
            \begin{align}
                rf_{\theta}=\dfrac{D\left(rV_{\theta}\right)}{Dt}
            \end{align}

        Khi tính toán phương trình năng lượng cho lưu chất, chúng ta dẫn đến một đại lượng được gọi là \bi{rothalpy} mà định nghĩa của nó được cho bởi:
            \begin{align}
                I=h+\dfrac{W^2}{2}-\dfrac{U^2}{2},
            \end{align}
        trong đó $U$ là vận tốc tuyệt đối của lưu chất và $W$ là vận tốc tương đối của lưu chất so với hệ quy chiếu quay: $\underline{W}=\underline{U}-\underline{\omega}\wedge\underline{r}$.

        Đối với một dòng chuyển động không nén được, biến thiên enthalpy được tính:
            \begin{align}
                \Delta h=\dfrac{1}{2}\left(U_2^2-U_1^2\right)+\dfrac{1}{2}\left(W_2^2-W_1^2\right).
            \end{align}
        Theo đó mức biến đổi áp suất tĩnh được tính theo công thức khi có sự mất mát áp suất đối với dòng chuyển động không nén được:
            \begin{align}
                p_2-p_1=\dfrac{1}{2}\left(W_1^2-W_2^2\right)-\Delta p_{\text{mất mát}}
            \end{align}

        Đối với một lưu chất nén được, không thể đơn giản lấy hiệu số hai trạng thái mà phải thực hiện một phép tính tích phân, một phép tính sự biến thiên enthalpy cho ta:
            \begin{align}
                \Delta h=h_2-h_1=\int_1^2\dfrac{dp}{\rho}+\int_1^2Tds
            \end{align}
        trong đó ta đã sử dụng hệ thức liên hệ enthalpy cho lưu chất nén được: $dh=dp/\rho+Tds$ và sự mất mát của dòng chuyển động đoạn nhiệt đã bao gồm trong sự tăng entropy. Nếu không có sự mất mát thì entropy là hằng số. Ta kí hiệu áp suất sau khi tăng là $p_2$ nếu có sự mất mát và $p_{2s}$ nếu không có sự mất mát.

        Một phần sự tăng enthalpy tĩnh và áp suất tĩnh đến từ số hạng $1/2\cdot\left(U_2^2-U_1^2\right)$, số hạng này không gắn với sự mất mát. Đối với một máy nén hướng trục, chúng ít khi đối mặt với hiệu tượng tách rời lớp biên, do đó chúng dược sử dụng với hai lý do: nếu phần lớn sự tăng enthalpy tĩnh đóng góp đến sự thay đổi vận tốc cánh chong chóng giữa nhập lượng và xuất lượng, áp suất tăng kỳ vọng sẽ thu được và hiệu suất sẽ đủ cao thậm chí là hành vi khí động lực học là nghèo nàng trong một vùng dòng chuyển động tách rời lớp biên rộng. Thực tế, giới hạn áp suất tăng cực đại phụ thuộc vào độ bền của vật liệu được dùng làm bánh công tác.

        Sự biến đổi bán kính có một ảnh hưởng quan trọng đối với máy hướng trục. Thực tế, một sự biến đổi nhỏ trong bán kính tạo ra một sự biến đổi "tự do" của enthalpy với một bậc độ lớn giống với bậc độ lớn của nó sinh bởi sự lệch và sự giảm tốc của dòng chuyển động trên cánh. Sự biến đổi là "tự do" vì nó thay đổi mà không có sự mất mát và không tạo ra sự tách rời lớp biên. Điều này dẫn đến một hiệu ứng ở bầu mà ta sẽ diển giải như sau:
        \begin{itemize}
            \item Nếu bán kính bầu tăng và bán kính thành là hằng số thì hiệu ứng bầu đã được sử dụng.
            \item Nếu bán kính bầu là không đổi và bán kính thành thay đổi thì hiệu ứng bầu đã không được sử dụng.
        \end{itemize}

        Đối với một dòng chuyển động trong một máy hướng trục, khoảng cách của đường dòng cách trục thay đổi đáng kể và do đó có thể tính đến sự biến đổ rothalpy khi xét đại lượng $T-U^2/2c_p$, đôi khi được gọi là sự giảm nhiệt độ (và áp suất liên kết) hơn là sự biến đổi nhiệt độ và áp suất tĩnh. Phương trình chuyển động dọc theo ống dòng tức thời có thể được viết:
            \begin{align}
                \dfrac{\partial V}{\partial t}+V\dfrac{\partial V}{\partial s}=-\dfrac{1}{\rho}\dfrac{\partial p}{\partial s}
            \end{align}
        và đố với enthalpy tĩnh:
            \begin{align}
                \dfrac{\partial h_0}{\partial s}=\dfrac{\partial h}{\partial s}+\dfrac{1}{2}\dfrac{\partial V^2}{\partial s}
            \end{align}
        khi sử dụng quan hệ nhiệt động lực học $Tds=dh-dp/\rho$ và xét quá trình đoạn nhiệt, ta có:
            \begin{align}
                \dfrac{\partial h_0}{\partial s}=\dfrac{1}{\rho}\dfrac{\partial p}{\partial s}+\dfrac{1}{2}\dfrac{\partial V^2}{\partial s}
            \end{align}
        kết hợp những phương trình này, ta thu được phương trình cho dòng chuyển động đoạn nhiệt thuận nghịch:
            \begin{align}
                \dfrac{\partial h_0}{\partial s}=-\dfrac{\partial V}{\partial t}
            \end{align}
        Tuy nhiên điều này dẫn đến một viễn cánh là nếu dòng chuyển động là ổn định, không có công nhập lượng. Do đó, sau một vài phép biến đổi, ta viết lại được phương trình này dưới dạng:
            \begin{align}
                \dfrac{Dh_0}{Dt}=\dfrac{1}{\rho}\dfrac{\partial p}{\partial t}
            \end{align}
        Điều này dẫn đến việc enthalpy dừng là không đổi nếu dòng chảy là ổn định.
\section{Tỉ lệ động lực học}
\subsection{Tỉ lệ hình học}
        Đây là dạng tỉ lệ đơn giản nhất, bao gồm độ đặc, tỉ lệ bình diện, tỉ lệ bầu. Mặc dù tỉ lệ hình học là đơn giản, đôi khi không phải bao giờ việc chọn một tỉ lệ hình học cũng phù hợp.
\subsection{Tỉ lệ khí động học toàn cục}
        Sự lựa chọn tỉ lệ khí động học phụ thuộc nhiều vào thông tin được quan tâm, có thể là lưu lượng khối, tỉ lệ áp suất, tốc độ quay và hiệu suất. 

        Tỉ lệ áp suất là một tham số vô thứ nguyên nhưng nó quan trọng bởi vì nó sẽ cho biết khả năng của một máy nén. Ta chỉ quan tâm đến tỉ lệ áp suất toàn phần và nó được tính nhờ vào sự hiệu chỉnh số Mach.

        Tỉ lệ áp suất và lưu lượng khối phụ thuộc vào tốc độ quay mà ta có thể chuyển tốc độ quay này về một số vô thứ nguyên khi lấy tỉ lệ nó với vận tốc âm thanh, nếu chỉ có một loại khí đi vào bên trong máy nén ta sẽ có $\gamma R$ là hằng số do đó tỉ số $U/\sqrt{T_0}$ (trong đó $T_0$ là nhiệt độ tù hãm của lưu chất ở đầu vào). Nếu bán kính ở mũi là hằng số thì tốc độ góc quay là đủ, ta sẽ sử dụng $N/\sqrt{T_0}$. Tuy nhiên, để thuận tiện hơn, người ta sẽ diển tả tốc độ quay thông qua hai biến số hiệu chỉnh là:
            \begin{align}
                N_{\text{hiệu chỉnh}}=\dfrac{N}{\sqrt{\theta}}=N\sqrt{\dfrac{T_{0\text{tham chiếu}}}{T_0}}
            \end{align}
        Nhiệt độ $T_{0\text{tham chiếu}}$ là tham chiếu, thường là nhiệt độ ở cao độ mặt biển tiêu chuẩn. 

        Lưu lượng khối lượng là một tham số hiệu suất quan trọng bởi vì 


\end{document}